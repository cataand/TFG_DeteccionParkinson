\capitulo{Introducción}
\label{cha:Introducción}

La enfermedad de Parkinson es una enfermedad neurodegenerativa que afecta a
millones de personas en todo el mundo. Se caracteriza por la degradación de las
células nerviosas que controlan el movimiento, lo que puede provocar temblores,
rigidez muscular y pérdida del control postural \cite{eswiki:148845196}. Aunque
no existe una cura para la enfermedad de Parkinson, la detección temprana y el
tratamiento adecuado pueden mejorar significativamente la calidad de vida de los
pacientes.

En los últimos años, la inteligencia artificial se ha convertido en una
herramienta de gran utilidad para el diagnóstico y tratamiento de enfermedades,
incluida la enfermedad de Parkinson. En este proyecto, se ha desarrollado un
sistema basado en inteligencia artificial para crear modelos de aprendizaje
automático capaces de detectar si un individuo padece la enfermedad de Parkinson
a partir de un vídeo de la persona realizando un gesto característico con la
mano, además de cierta información adicional sobre el individuo.

Interactuar con los modelos creados de forma directa puede resultar complicado
si no se dispone de ciertos conocimientos previos sobre programación e
inteligencia artificial. Con el objetivo de facilitar esta tarea para la persona
media se ha desarrollado una aplicación web que simplifica el uso de los
modelos.

Los resultados obtenidos indican que los modelos creados son capaces de detectar
la enfermedad de Parkinson con bastante precisión, lo que sugiere que podrían
ser de utilidad para la detección de la enfermedad. Además, el enfoque basado en
vídeo es no invasivo y fácil de usar, lo que lo hace potencialmente útil en
entornos clínicos y de telemedicina.
