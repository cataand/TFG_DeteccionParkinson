\capitulo{Objetivos del proyecto}
\label{cha:Objetivos del proyecto}

\section{Objetivos generales}

El objetivo principal de este proyecto es la creación de un sistema basado en la
inteligencia artificial que permita la detección de bradicinesia de forma
consistente utilizando como entrada un vídeo de la prueba de <<finger-tapping>>
(además de otra información), además este sistema debería ser utilizable por la
persona media. Para alcanzar este objetivo se plantean los siguientes objetivos
específicos:

\begin{enumerate}
    \item Revisar los trabajos previos relacionados para obtener unos
    conocimientos iniciales sobre el tema e identificar qué métodos funcionan
    mejor y peor.
    \item Diseñar e implementar una secuencia de pasos o <<pipeline>> para
    extraer características relevantes de los vídeos para su uso en algoritmos
    de aprendizaje automático.
    \item Buscar un modelo que se ajuste considerablemente bien a los datos
    disponibles mediante alguna de las técnicas existentes en el campo del
    aprendizaje automático para este propósito.
    \item Crear una aplicación web que permita subir un vídeo de la prueba,
    extraer una serie de características de este y proporcionárselas a un modelo
    de aprendizaje automático que sea capaz de devolver una predicción relativa
    a la presencia o no de bradicinesia en el paciente.
    \item Debido al público de una aplicación de este tipo se debe tener muy en
    cuenta la accesibilidad a la hora de diseñarla.
    \item Incluir en la aplicación web la funcionalidad de gestión de usuarios,
    así como la posibilidad de actualizar el modelo utilizado para realizar las
    predicciones por parte de un administrador.
    \item Documentar el trabajo realizado.
\end{enumerate}

\section{Objetivos técnicos}

Los siguientes objetivos han están relacionados con los aspectos técnicos de la
aplicación.

\begin{itemize}
    \item Utilizar Docker para facilitar el despliegue de la
    aplicación en cualquier equipo que tenga el motor de Docker instalado.
    \item Crear una API (Application Programming Interface) accesible mediante
    internet desde cualquier máquina para separar de forma clara el <<backend>>
    y el <<frontend>> y permitir la extensión futura de la aplicación.
    \item Internacionalizar la aplicación para permitir el uso de esta en varios
    idiomas y realizar una detección automática del idioma preferido por el
    usuario.
\end{itemize}