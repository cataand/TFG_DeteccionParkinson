\capitulo{Trabajos relacionados}
\label{cha:Trabajos relacionados}

Durante la última década se ha intentado utilizar la visión por computador para
evaluar la Enfermedad de Parkinson en múltiples ocasiones con diferentes
resultados. Este capítulo recopila de forma resumida algunos de los trabajos más
relevantes.


\section{A computer vision framework for finger-tapping evaluation}

Este artículo~\cite{khan2014computer} documenta el uso de visión por computador
para determinar el nivel de severidad de la Enfermedad de Parkinson y distinguir
entre individuos con esta enfermedad e individuos sin ella.

El método empleado se caracteriza por utilizar vídeos con la cara de la persona
y ambas manos a los lados de la cabeza, apuntando las puntas de los dedos hacia
la misma. Esto se utiliza para poder normalizar las distancias en base a
características faciales.

El estudio se realizó sobre 13 pacientes con la Enfermedad de Parkinson, tomando
17 vídeos de cada paciente durante un día, además de un grupo de control de 6
individuos, tomando 2 vídeos al día durante una semana por cada uno. Aunque
algunos de estos vídeos fueron descartados. En total se utilizaron 471 vídeos.


\paragraph{Metodología}

\begin{enumerate}
    \item Se detecta la cara del individuo para la normalización. Esto se basa
          en que la longitud de la mano de una persona adulta es aproximadamente
          igual a la altura de su cara.
    \item Se obtiene una serie temporal que representa la amplitud del
          movimiento de los dedos índice y pulgar de la mano dominante del
          individuo.
    \item Se extraen un total de 15 características de esta serie temporal, por
          ejemplo:
          \begin{itemize}
              \item Correlación cruzada media entre los máximos locales de dos
                    intervalos distintos de tiempo de la serie temporal. Esto
                    mismo se realiza también sobre los mínimos locales.
              \item Número total de toques de dedos durante la grabación.
              \item Velocidad media de la apertura de dedos.
              \item Velocidad media del cierre de dedos.
              \item \dots
          \end{itemize}
    \item Se realiza una selección de características eliminando aquellas
          redundantes y usando el algoritmo chi-cuadrado.
    \item Se entrena una máquina de vectores de soporte (SVM) mediante las
          características obtenidas para realizar la clasificación.
\end{enumerate}


\paragraph{Resultados}

En cuanto a la distinción entre pacientes de la Enfermedad de Parkinson y el
grupo de control se obtuvo una precisión del 95\%, que es una cifra que se
debería tomar con precaución debido que, aunque se han utilizado 471 vídeos,
estos provienen de únicamente 19 personas.


\section{The discerning eye of computer vision}

En este estudio \cite{williams2020discerning} realizado sobre 39 pacientes con
la Enfermedad de Parkinson y sobre un grupo de control de 30 individuos se
tomaron vídeos de ambas manos de cada individuo mientras realizan toques de los
dedos índice y pulgar (de forma similar a cómo se han tomado las muestras para
este trabajo). Dando un total de 133 vídeos (se descartó uno).

De estos vídeos se han extraído diferentes características y comprobado la
relación que existe entre éstas y diferentes escalas que clasifican el nivel de
gravedad de la Enfermedad en un paciente, como la \textit{Modified Bradykinesia
    Rating Scale} (MBRS) que categoriza el movimiento de los individuos en 5
niveles, del 0 al 4, siendo 0 un movimiento normal y 4 el nivel de mayor
gravedad.


\paragraph{Metodología}

\begin{enumerate}
    \item Se utiliza una librería de visión por computador, en concreto
          DeepCutLab, para obtener una serie temporal de la amplitud entre las
          puntas de los dedos pulgar e índice.
    \item Se normaliza esta serie temporal utilizando la amplitud máxima
          detectada, que va a convertirse en el valor 1, siendo todos los demás
          valores escalados proporcionalmente.
    \item Se extraen las siguientes características:
          \begin{itemize}
              \item Velocidad, calculada como la tasa media de cambio.
              \item Variabilidad de la amplitud, calculada como el coeficiente
                    de variación de la diferencia media entre máximos y mínimos
                    de diferentes intervalos de 1 segundo de la serie temporal.
              \item Regularidad del ritmo, calculada utilizando la Transformada
                    Rápida de Fourier y, a continuación, midiendo la potencia de
                    la frecuencia dominante más la potencia de las frecuencias
                    en un intervalo de 0.4 Hz alrededor de ésta (un ritmo más
                    regular concentra una mayor potencia en una única
                    frecuencia).
          \end{itemize}
\end{enumerate}


\paragraph{Resultados}

Se observó una correlación bastante alta entre las características utilizadas y
la categoría del individuo dentro de las escalas de medición de la Enfermedad de
Parkinson utilizadas medida por un experto en el campo.


\section{Supervised classification of bradykinesia}

Este estudio \cite{williams2020supervised} es muy similar al anteriormente
explicado, y está realizado por un equipo compuesto por casi los mismos
participantes. En este caso se utilizaron 70 vídeos, de ambas manos de 20
pacientes con la Enfermedad de Parkinson y de un grupo de control de 15
individuos.


\paragraph{Metodología}

La metodología es prácticamente igual que antes, la diferencia principal está en
las características que se extraen de la serie temporal correspondiente con la
amplitud, se ha obtenido:

\begin{itemize}
    \item Frecuencia, medida como la frecuencia máxima de la Transformada Rápida
          de Fourier de la serie temporal.
    \item Amplitud, calculada como la densidad espectral, que se ha obtenido
          mediante la integral cuadrada del espectro de la Transformada Rápida
          de Fourier.
\end{itemize}

Con estas características se ha realizado clasificación binaria mediante
clasificación bayesiana ingenua (naive bayes), regresión logística y máquina de
vectores de soporte, tanto con función lineal como con función de base radial.


\paragraph{Resultados}

Los mejores resultados se obtuvieron con máquina de vectores de soporte con
función de base radial, que coincide en un 73\% de los casos con la
clasificación de expertos en el campo.
