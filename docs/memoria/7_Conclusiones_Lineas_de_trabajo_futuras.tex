\capitulo{Conclusiones y Líneas de trabajo futuras}
\label{cha:Conclusiones y Líneas de trabajo futuras}

En el campo de la minería de datos la fase de preprocesado tiene tanta
importancia como el propio entrenamiento de los modelos. En esta fase se
determina lo complejos que deberán ser los modelos para adaptarse a los datos y
el tiempo de entrenamiento que será necesario.

\section{Lineas de trabajo futuras}

El conjunto de datos utilizado contine muestras de personas que ya se conoce que
padecen la enfermedad de Parkinson, debido a esto, en este proyecto se ha
conseguido detectar la presencia de la enfermedad una vez está en un estado
avanzado. Esto no es muy útil para realizar una detección temprana y empezar un
tratamiento más efectivo en mermar los efectos de la enfermedad.

Una línea de trabajo futuro a largo plazo podría consistir en tomar vídeos de la
prueba de <<finger-tapping>> de forma aleatoria sobre la población, si la
muestra es lo suficientemente grande, algunas de esas personas desarrollarán la
enfermedad. Con esto podría ser posible crear modelos capaces de detectar
personas que van a padecer la enfermedad de Parkinson en el futuro, pudiendo
tomar las medidas precautorias necesarias con antelación.
