\capitulo{Conclusiones y Líneas de trabajo futuras}
\label{cha:Conclusiones y Líneas de trabajo futuras}

Por último, esta sección recopila las conclusiones a las que se ha llegado
durante la realización del proyecto. Además de algunas líneas de trabajo
futuras.

\section{Conclusiones}

Se han extraído las siguientes conclusiones durante el proyecto.

\begin{itemize}
    \item En el campo de la minería de datos la fase de preprocesado tiene tanta
    importancia como el propio entrenamiento de los modelos. En esta fase se
    determina lo complejos que deberán ser los modelos para adaptarse a los
    datos y el tiempo de entrenamiento que será necesario.
    \item Se ha creado una librería de Python instalable mediante \texttt{pip}
    para centralizar el código común la fase de investigación (generación de
    modelos) y la de desarrollo (aplicación web). Esta librería contiene todo el
    código relacionado con el procesado de los vídeos para transformarlos a un
    conjunto de características.
    \item Para facilitar la interacción con los modelos generados por cualquier
    programa, siempre que pueda realizar peticiones HTTP, se ha desarrollado una
    API Rest mediante el framework FastAPI.
    \item Se ha creado una aplicación web mediante un framework de JavaScript
    (SvelteKit) para permitir a cualquier persona que disponga de un navegador
    web interactuar con la API.
    \item Se ha utilizado Docker para crear contenedores con las distintas
    partes que componen la aplicación, simplificando en gran medida el proceso
    de configuración del entorno de desarrollo y producción.
    \item Con el objetivo de agilizar el proceso de creación de modelos se ha
    implementado un <<script>> de Python que presenta una interfaz por línea de
    comandos mediante la que se determinan distintos parámetros para crear un
    modelo que se puede subir a la aplicación web.
\end{itemize}

\section{Líneas de trabajo futuras}

Algunas líneas de trabajo futuro a corto o medio plazo:

\begin{itemize}
    \item Buscar alguna forma para acelerar el procesado de los vídeos,
    actualmente esto toma unos 20 segundos, que es una espera muy larga al
    utilizar la aplicación web. Una idea que no se llegó a implementar es el uso
    de programación concurrente para <<aprovechar>> el tiempo de espera de
    entrada/salida existente cuando se leen los fotogramas del disco.
    \item Añadir la posibilidad de entrenar modelos directamente desde la
    aplicación web, eliminando la necesidad de configurar el entorno para esto
    de forma local.
    \item Permitir el uso de múltiples modelos distintos desde el formulario
    para obtener predicciones, posiblemente como un menú de <<Opciones
    avanzadas>>.
    \item Añadir algún método de actualización de token de sesión JWT para
    impedir que el usuario pierda su sesión mientras usa la aplicación web.
\end{itemize}

El conjunto de datos utilizado contine muestras de personas que ya se conoce que
padecen la enfermedad de Parkinson, debido a esto, en este proyecto se ha
conseguido detectar la presencia de la enfermedad una vez está en un estado
avanzado. Esto no es muy útil para realizar una detección temprana y empezar un
tratamiento más efectivo en mermar los efectos de la enfermedad.

Una línea de trabajo futuro a largo plazo podría consistir en tomar vídeos de la
prueba de <<finger-tapping>> de forma aleatoria sobre la población, si la
muestra es lo suficientemente grande, algunas de esas personas desarrollarán la
enfermedad. Con esto podría ser posible crear modelos capaces de detectar
personas que van a padecer la enfermedad de Parkinson en el futuro, pudiendo
tomar las medidas precautorias necesarias con antelación.
