\capitulo{Conceptos teóricos}
\label{cha:Conceptos teóricos}

Este capítulo define algunos de los conceptos teóricos que se mencionan en este
documento.

\subsection{Preprocesado}

La información con la que se entrene un modelo de aprendizaje automático
determina en gran medida el rendimiento que podrá alcanzar, debido a esto es muy
frecuente realizar un paso previo a la creación del modelo, denominado
\textit{preprocesado}.

El objetivo del preprocesado es transformar la entrada de datos iniciales con el
objetivo de permitir y facilitar que un modelo se adapte a los mismos.

El preprocesado también tiene una gran influencia sobre el tiempo de computación
necesario para entrenar un modelo y sobre la complejidad que necesitará para
adaptarse a los datos.

El preprocesado suele componerse de algunos de los siguientes pasos.

\subsubsection{Extracción de características}

La extracción de características es el proceso de identificar, seleccionar y
transformar atributos relevantes de los datos de entrada para su uso en un
modelo. Por ejemplo, en este proyecto, se han extraido características como la
velocidad de movimiento o amplitud a partir de un vídeo.

Existen diversas técnicas para la extracción de características, incluyendo las
selección manual de características, o técnicas automatizadas como la reducción
de dimensionalidad, la extracción de características basada en redes neuronales
\cite{intrator1991feature}, entre otras. La elección de la técnica de extracción
de características depende del conjunto de datos, del problema específico de
aprendizaje automático que se está abordando y del tipo de modelo de aprendizaje
automático que se está utilizando.

\subsubsection{Selección de características}

En general, los datos de entrada pueden ser muy complejos y estar compuestos por
una gran cantidad de información redundante o no relevante para el modelo. La
selección de características se utiliza para identificar  las características
más relevantes y representativas de los datos, que pueden ser utilizadas para
entrenar modelos de aprendizaje automático con mayor eficacia.

En la selección de características, se pueden utilizar técnicas como el análisis
de componentes principales (PCA) \cite{mackiewicz1993principal}, el análisis
discriminante lineal (LDA) \cite{xanthopoulos2013linear} o pruebas de
significancia.

\subsubsection{Normalización}

Existen modelos muy sensibles a que existan diferencias en la escala de los
distintos atributos, como, por ejemplo, \textit{k-nearest neighbors}, por lo que
es muy habitual que en la fase de preprocesado se realize una normalización de
los datos, es decir, transformarlos de tal forma que utilicen la misma escala,
en general se suelen tomar valores en los intervalos $[0, 1]$ o $[-1, 1]$.

Aunque lo más habitual es que la normalización se haga sin distorsionar las
diferencias entre los valores previos, existen situaciones en las que puede ser
ventajoso utilizar un método de normalización que sí altere estas diferencias,
por ejemplo, la normalización por cuantiles \cite{enwiki:1138433182}, en la que
se modifican los valores para que sigan una distribución normal, lo que consigue
que, si existen valores muy lejanos a los valores más comunes
(\textit{outliers}), estos se acerquen al resto. 
