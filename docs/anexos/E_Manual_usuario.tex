\capitulo{Documentación de usuario}
\label{cha:Documentación de usuario}

\section{Introducción}



\section{Requisitos de usuarios}

Para utilizar la aplicación el único requisito es disponer de una versión
reciente (a ser posible de los últimos dos años) de un navegador, programa que
suele venir por defecto en la mayoría de sistemas operativos. Además, cabe
destacar que la ejecución de JavaScript deberá estar permitida, aunque esto no
debería suponer un problema, al tener que ser desacivada de forma manual.

\section{Instalación}

Como la aplicación está alojada en un servidor remoto, el usuario no necesita
instalar ninguna dependencia específica.

\section{Manual del usuario}

Desde un navegador se puede acceder a la aplicación mediante la url
\href{https://paddle.catalin.sh}{paddel.catalin.sh}.

\subsection{Formulario predicción}

La página principal consiste en un formulario desde el que se pueden subir los
datos necesarios para realizar una predicción mediante los modelos:

\begin{itemize}
    \item Mano dominante
    \item Mano mostrada en el vídeo
    \item Edad
    \item Sexo
    \item Archivo de vídeo
\end{itemize}

\imagenConBorde{manual_usuario/01.png}{Formulario -- Página principal}{1}

Una vez introducidos los datos se puede hacer click en el botón <<Enviar>>, si
faltase algún dato el usuario es notificado por el navegador. Esto comenzará el
proceso de envío de los datos, debido al tamaño relativamente grande que tienen
los archivos de vídeo, este proceso puede ser de una duración bastante larga.
Debido a eso el usuario pasa momentaneamente a una pantalla de carga antes de
mostrar el resultado.

\imagenConBorde{manual_usuario/02.png}{Pantalla de carga formulario -- Subida}{1}

Cuando finalice la subida de los datos dará comienzo el procesado de los mismos
en el servidor, este, de nuevo, tiene una duración relativamente larga (entre 20
y 30 segundos), por lo que se informa al usuario para que no salga de la página
antes de tiempo.

\imagenConBorde{manual_usuario/03.png}{Pantalla de carga formulario -- Procesado}{1}

Cuando el proceso anterior finaliza el servidor responde con el resultado de la
predicción.

\imagenConBorde{manual_usuario/04.png}{Resultado predicción}{1}

En caso de que se haya producido algún error en la petición (problemas de red,
archivo de vídeo inválido, etc.) se notifica al usuario mediante una pantalla de
error.

\imagenConBorde{manual_usuario/05.png}{Error predicción}{1}

\subsection{Cambio de idioma}

Acceder a \href{https://paddle.catalin.sh}{paddel.catalin.sh} leerá el idioma
preferido del navegador usado y redirigirá a la alternativa más adecuada de
entre los idiomas disponibles (es-ES y en-GB).

El usuario puede controlar el idioma utilizado mediante el selector de idioma
localizado en la parte superior derecha de todas las páginas de la aplicación.

\imagenConBorde{manual_usuario/06.png}{Selector de idiomas}{0.2}

Este menú, además de cambiar el idioma en la sesión actual, utiliza una variable
en
\href{https://developer.mozilla.org/es/docs/Web/API/Window/localStorage}{almacenamiento
local} para \textit{recordar} el idioma seleccionado y utilizarlo en posteriores
visitas a la aplicación.

\subsection{Administración}

Para acceder al panel de administración se utiliza el botón <<Administración>>
de la página principal, si el usuario tiene una sesión válida abierta, será
redirigido a la página de administración de modelos, en caso contrario será
llevado a la página de inicio de sesión.

\imagenConBorde{manual_usuario/07.png}{Formulario de inicio de sesión}{0.6}

Desde esta página un usuario administrador puede identificarse introduciendo su
nombre de usuario y su contraseña.

En caso de que los datos introducidos sean incorrecto se le notifica con un
mensaje de error.

\imagenConBorde{manual_usuario/08.png}{Selección de modelo}{0.6}

\subsubsection{Administración de modelos}

En este panel se pueden ver y gestionar los modelos disponibles en la
aplicación.

\paragraph{Seleccionar modelo}

Para seleccionar el modelo a utilizar para realizar predicciones se puede
utilizar el botón <<Seleccionar>> correspondiente.

\imagenConBorde{manual_usuario/09.png}{Error inicio sesión}{1}

\paragraph{Añadir modelo}

Añadir un nuevo modelo se puede realizar mediante el botón <<Añadir modelo>>,
que abre una ventana modal con un formulario para añadir la información del
nuevo modelo.
