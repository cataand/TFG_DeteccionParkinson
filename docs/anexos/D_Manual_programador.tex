\chapter{Documentación técnica de programación}
\label{cha:Documentación técnica de programación}

\section{Introducción}

Esta sección contiene toda la información que una persona externa debería tener
para poder trabajar con las diferentes partes de este projecto.

Existen varios archivos \texttt{Makefile} en diferentes lugares del proyecto que
contienen diferentes comandos útiles que se repiten con frecuencia.

\section{Estructura de directorios}

En la raíz del proyecto se pueden encontrar los siguentes directorios:

\subsection{\texttt{/app/}}

Proyecto en Docker que implementa los contenedores que conforman la aplicación
web. Este directorio contiene varios subdirectorios con el código fuente y
configuración de las diferentes partes de la aplicación.

\subsubsection{\texttt{/app/api/}}

Este directorio contiene la implementación de la API de la aplicación, se ha
realizado en Python mediante la librería
\href{https://fastapi.tiangolo.com/}{FastAPI}.

\subsubsection{\texttt{/app/proxy/}}

Configuración para el contenedor de Caddy que implementa un proxy inverso que
redirige las peticiones que llegan desde el exterior al contenedor apropiado
(API o web).

\subsubsection{\texttt{/app/web/}}

Proyecto de NextJS (framework de JavaScript) que implementa el \textit{frontend}
de la aplicación.

\subsection{\texttt{/docs/}}

Proyecto en \LaTeX{} que contiene este documento junto con la memoria.

\subsubsection{\texttt{/docs/anexos/}}

Contiene diferentes ficheros \texttt{.tex} que conforman estos anexos.

\subsubsection{\texttt{/docs/common/}}

Archivos \texttt{.tex} comunes entre la memoria y estos anexos.

\subsubsection{\texttt{/docs/img/}}

Diferentes imágenes utilizadas en la documentación.

\subsubsection{\texttt{/docs/memoria/}}

Ficheros \texttt{.tex} que contienen los diferentes apartados de la memoria.

\subsection{\texttt{/notebooks/}}

Notebooks de Jupyter que se han utilizado para realizar pruebas, entrenar
modelos, optimizar hiperparámetros y generar gráficas.

\subsection{\texttt{/paddel/}}

Librería PADDEL, es un proyecto de Python instalable mediante \texttt{pip} que
contiene el código utilizado para realizar la fase de investigación
(procesamiento de los vídeos, extracción de características, etc.) del que va a
depender la aplicación web.

\subsection{\texttt{/paddel/src/paddel/}}

Contiene los diferentes archivos de Python que componen la librería PADDEL.

\subsubsection{\texttt{/paddel/src/paddel/hyper\_parameters/}}

Contiene los archivos de Python relacionados con la fase de optimización de
hiperparámetros.

\subsubsection{\texttt{/paddel/src/paddel/preprocessing/}}

Contiene los archivos de Python relacionados con el preprocesado y
transformación de los vídeos para reducirlos a un conjunto de características.

\section{Manual del programador}

En esta sección se detalla todo lo que debería saber una persona que para
realizar cambios sobre las diferentes partes que componen este proyecto.

\subsection{Aplicación web}

Para poder utilizar esta aplicación Docker debe ser instalado con antelación. La
\href{https://docs.docker.com/engine/install/}{documentación de Docker} detalla
el proceso de instalación sobre diferentes plataformas. Una vez Docker está
instalado el proceso siguiente debería ser agnóstico al sistema operativo
utilizado.

La aplicación se compone por un conjunto de contenedores de Docker que
interactúan entre ellos. En la raíz de la aplicación (\texttt{/app/}) se
encuentran varios archivos \textit{docker-compose} de tipo YAML:

\begin{itemize}
    \item \texttt{docker-compose.yml}: Contiene la configuración básica común
          tanto para el entorno de desarrollo como para el de producción.
          Contiene información como dependencias entre contenedores, variables
          de entorno que se pasa a cada contenedor y puertos en los que se va a
          servir la aplicación.
    \item \texttt{docker-compose.dev.yml}: Fichero con la configuración de los
          contenedores específica al entorno de desarrollo. Simplemente
          establece un mapeo entre los directorios del equipo anfitrión y los de
          los contenedores para que los cambios realizados desde el anfitrión se
          vean reflejados dentro de los contenedores y éstos se actualicen de
          forma acorde.
    \item \texttt{docker-compose.prod.yml}: Fichero con la configuración de los
          contenedores específica al entorno de producción. Simplemente
          establece el reinicio automático de los contenedores, para que en caso
          de fallo o reinicio del sistema, los contenedores se lancen junto al
          servicio de Docker.
\end{itemize}

Gracias al uso de Docker Compose se crea una red interna de Docker que conecta
los diferentes contenedores entre sí y estableciendo \textit{hostnames} simples
que pueden utilizar para conectarse unos con otros. Por ejemplo, el proxy
inverso puede realizar una petición HTTP a la API en la ruta
\texttt{http://api}.

La configuración de los distintos contenedores se realiza mediante variables de
entorno. Con el fin de establecer estas variables se ha creado el fichero
\texttt{sample.env} que contiene unos valores básicos para estas variables que
deberán ser cambiadas para adaptarse al entorno en el que se van a ejecutar los
contenedores. Una vez realizados los cambios este fichero debe ser guardado con
el nombre \texttt{.env} en el mismo directorio donde se encuentra
\texttt{sample.env}. Este fichero \texttt{.env} es detectado de forma automática
por Docker cuando se lanzan los contenedores.

Para las operaciones de arranque y parada de los contenedores se utiliza el
comando \texttt{docker compose} junto con los parámetros adecuados para la
operación que se desea realizar. Como estos comando no son triviales y se
utilizan de forma frecuente se encuentran guardados en un fichero
\texttt{Makefile}
. Son posibles los siguientes comandos\footnote{Para utilizar un archivo
    \texttt{Makefile} se necesita la utilidad \texttt{make}, en caso de que no
    se disponga de la misma se pueden copiar los comandos del archivo
    \texttt{Makefile} y utilizar manualmente.}:

\begin{itemize}
    \item \texttt{make down}: Equivalente al comando:
          \begin{flushleft}
              \texttt{docker compose down ----remove-orphans ----rmi all ----volumes ----timeout 0}
          \end{flushleft}
          Elimina los contenedores y todo lo relacionado con los mismos sin
          esperar. No se elimina el caché que guarda Docker, por lo que si se
          vuelve a lanzar los contenedores no es necesario volver a descargar e
          instalar todo.

    \item \texttt{make dev}: Equivalente al comando:
          \begin{flushleft}
              \texttt{make down \&\& \\
                  docker compose -f docker-compose.yml \\
                  -f docker-compose.dev.yml up -d}
          \end{flushleft}
          Para y elimina los contenedores si estos estaban funcionando y los
          lanza en modo desarrollo.

    \item \texttt{make prod}: Equivalente al comando:
          \begin{flushleft}
              \texttt{make down \&\& \\
                  docker compose -f docker-compose.yml \\
                  -f docker-compose.prod.yml up -d}
          \end{flushleft}
          Para y elimina los contenedores si estos estaban funcionando y los
          lanza en modo producción.
\end{itemize}

Para el desarrollo de la aplicación, como es de esperar, se utiliza el comando
\texttt{make dev} para arrancar los contenedores (habiendo antes instalado
Docker y creado el archivo \texttt{.env}).

En modo desarrollo se pueden editar los ficheros directamente desde el sistema
anfitrión y los cambios se van a ver reflejados sobre los contenedores, esto
puede ser útil para realizar cambios pequeños en los que no se necesiten las
ayudas que puede dar una IDE.

Para utilizar un entorno integrado para realizar el desarrollo se deben utilizar
herramientas que pueden conectarse a contenedores de Docker, como, por ejemplo,
\href{https://code.visualstudio.com/}{Visual Studio Code} con la extensión
\href{https://marketplace.visualstudio.com/items?itemName=ms-vscode-remote.remote-containers}{Dev
    Containers} o IDEs de JetBrains, como IntelliJ o Pycharm, con el plugin de
Docker.

En los siguientes apartados se detalla el proceso de desarrollo sobre los
diferentes contenedores.

\subsubsection{Proxy}

El contenedor \textit{proxy} contiene una instancia de Caddy funcionando como
proxy inverso. Caddy es un servidor web similar a Nginx, pero con algunas
características adicionales, como la gestión automática de certificados SSL.

Los archivos que utiliza este contenedor se encuentran en la carpeta
\texttt{/app/proxy}:

\begin{itemize}
    \item \texttt{Dockerfile}: Contiene la imágen y el proceso a seguir que
          Docker debe realizar para construir el contenedor.
    \item \texttt{Caddyfile}: Contiene la configuración de redirecciones de
          Caddy que determina el contenedor de destino para las peticiones que
          recibe.
\end{itemize}

Este contenedor es el único que tiene acceso al exterior, por lo que todas las
peticiones que se hagan a la aplicación van a pasar por él.

\subsubsection{API}

El contenedor \textit{api} se trata de una implementación basada en la librería
FastAPI. Se puede obtener una visión general de las peticiones posibles en la
ruta \texttt{/api} de la localización en la que se encuentra el servicio.

El fichero \texttt{Dockerfile} define como se prepara el contenedor, los pasos
generales que sigue son:

\begin{enumerate}
    \item Copiar (o montar) el código fuente de la API y la librería PADDEL
          desde el anfitrión.
    \item Instalar los requisitos (\texttt{requirements.txt}), que incluyen
          PADDEL y las dependencias específicas de la API.
    \item Se lanza el servicio de FastAPI con los parámetro adecuados
          dependiendo de si se lanza en modo desarrollo o producción.
\end{enumerate}

\subsubsection{Web}

El contenedor \textit{web} es un proyecto basado en NodeJS que utiliza el
framework de JavaScript NextJS, este framework está basado en React, una
librería de JavaScript que permite la creación de componentes interactivos y
reutilizables dentro de una web. Debido a esto, la
\href{https://nextjs.org/docs/getting-started}{documentación de NextJS} es un
recurso de gran valor para el desarrollo de este proyecto.

En la raíz del proyecto existen varios archivos de interés:

\begin{itemize}
    \item \texttt{Dockerfile}: Define cómo se crea el contendor.
    \item \texttt{next.config.js}: Define algunas configuraciones para NextJS.
    \item \texttt{package.json}: Especifica detalles sobre el proyecto de NodeJS
          como nombre del proyecto, autor, dependencias y algunos comandos de
          utilidad.
    \item \texttt{package.lock.json}: Este archivo define las versiones
          específicas que se han utilizado de cada dependencia junto con sumas
          de comprobación para asegurar que se puede reproducir el entorno en el
          que se va a ejecutar la aplicación de forma exacta.
    \item \texttt{postcss.config.js}, \texttt{tailwind.config.js}: Archivos
          utilizados para la configuración de Tailwind CSS.
    \item \texttt{tsconfig.json}: Archivo que configura el compilador de
          TypeScript.
    \item \texttt{.eslint.json}: Archivo de configuración de ESLint.
\end{itemize}

Aunque los archivos anteriores son importantes, la mayoría de cambios y
adiciones se realizan dentro de los directorios \texttt{/pages/}, que define la
estructura de páginas de la web, y \texttt{/components/}, donde se encuentran
los diferentes componentes reutilizables de la web.

Los archivos estáticos que se utilizan, imágenes, vídeos, iconos, se almacenan
en el directorio \texttt{/public/}.

En el directorio \texttt{/styles/} se alojan los archivos que definen los
estilos (CSS) de la web, el contenido de esta carpeta es bastante escaso debido
al extenso uso de las clases que provee la librería Tailwind CSS.

\subsection{Documentación}

La documentación se trata de un proyecto de \LaTeX{} compuesto por varios
archivos \texttt{.tex} que se compilan a archivos \texttt{.pdf}.

Para poder compilar estos archivos se necesita tener instalada una distribución
de \LaTeX{}, en general la más recomendada es
\href{https://www.tug.org/texlive/}{TeX Live}, disponible en varias plataformas.

Una vez instalado \LaTeX{} se debería tener acceso al comando \texttt{latexmk},
que se utiliza dentro del archivo \texttt{Makefile} tanto para compilar los
archivos como para limpiar los archivos sobrantes. Están definidos los
siguientes comandos:

\begin{itemize}
    \item \texttt{make memoria}: Compila la memoria. Equivalente a ejecutar:
          \begin{flushleft}
              \texttt{latexmk -cd -pdf memoria.tex}
          \end{flushleft}
    \item \texttt{make anexos}: Compila los anexos. Equivalente a ejecutar:
          \begin{flushleft}
              \texttt{latexmk -cd -pdf anexos.tex}
          \end{flushleft}
    \item \texttt{make all}: Compila la memoria y los anexos. Equivalente a
          ejecutar los comandos anteriores.
    \item \texttt{make clean}: Elimina archivos auxiliares generados durante la
          compilación. Equivalente a ejecutar:
          \begin{flushleft}
              \texttt{latexmk -cd -pdf -bibtex-cond1 -c memoria.tex} \\
              \texttt{latexmk -cd -pdf -bibtex-cond1 -c anexos.tex}
          \end{flushleft}
\end{itemize}

Para editar los archivos de este proyecto existen algunas alternativas, como el
entorno integrado creado específicamente para \LaTeX{}
\href{https://www.xm1math.net/texmaker/}{Texmaker} o Visual Studio Code con la
extensión
\href{https://marketplace.visualstudio.com/items?itemName=James-Yu.latex-workshop}{Latex
    Workshop}.

\subsection{Librería PADDEL}

\section{Compilación, instalación y ejecución del proyecto}


\section{Pruebas del sistema}
