\capitulo{Documentación técnica de programación}
\label{cha:Documentación técnica de programación}

\section{Introducción}

Esta sección contiene toda la información que una persona externa debería tener
para poder trabajar con las diferentes partes de este projecto.

Existen varios archivos \texttt{Makefile} en diferentes lugares del proyecto que
contienen diferentes comandos útiles que se repiten con frecuencia.

\section{Estructura de directorios}

En la raíz del proyecto se pueden encontrar los siguentes directorios:

\newpage

\dirtree{%
    .1 /app/ \\
    \hphantom{0cm}{}
    \begin{minipage}[t]{10cm}
        \normalfont
        Proyecto en Docker que implementa los contenedores que conforman la aplicación
        web{.} Este directorio contiene varios subdirectorios con el código fuente y
        configuración de las diferentes partes de la aplicación{.}
    \end{minipage}.
    .2 api/ \\
    \hphantom{0cm}{}
    \begin{minipage}[t]{10cm}
        \normalfont
        Este directorio contiene la implementación de la API de la aplicación, se ha
        realizado en Python mediante la librería
        \href{https://fastapi.tiangolo.com/}{FastAPI}{.}
    \end{minipage}.
    .2 proxy/ \\
    \hphantom{0cm}{}
    \begin{minipage}[t]{10cm}
        \normalfont
        Configuración para el contenedor de Caddy que implementa un proxy inverso que
        redirige las peticiones que llegan desde el exterior al contenedor apropiado
        (API o web){.}
    \end{minipage}.
    .2 web/ \\
    \hphantom{0cm}{}
    \begin{minipage}[t]{10cm}
        \normalfont
        Proyecto de SvelteKit (framework de JavaScript) que implementa el
        \textit{frontend} de la aplicación{.}
    \end{minipage}.
}

\vspace{1cm}

\dirtree{%
    .1 /docs/ \\
    \hphantom{0cm}{}
    \begin{minipage}[t]{10cm}
        \normalfont
        Proyecto en \LaTeX{} que contiene este documento junto con la memoria{.}
    \end{minipage}.
    .2 anexos/ \\
    \hphantom{0cm}{}
    \begin{minipage}[t]{10cm}
        \normalfont
        Contiene diferentes ficheros \texttt{.tex} que conforman estos anexos{.}
    \end{minipage}.
    .2 common/ \\
    \hphantom{0cm}{}
    \begin{minipage}[t]{10cm}
        \normalfont
        Archivos \texttt{.tex} comunes entre la memoria y estos anexos{.}
    \end{minipage}.
    .2 img/ \\
    \hphantom{0cm}{}
    \begin{minipage}[t]{10cm}
        \normalfont
        Diferentes imágenes utilizadas en la documentación{.}
    \end{minipage}.
    .2 memoria/ \\
    \hphantom{0cm}{}
    \begin{minipage}[t]{10cm}
        \normalfont
        Ficheros \texttt{.tex} que contienen los diferentes apartados de la memoria{.}
    \end{minipage}.
}

\newpage

\dirtree{%
    .1 /notebooks/ \\
    \hphantom{0cm}{}
    \begin{minipage}[t]{10cm}
        \normalfont
        Notebooks de Jupyter que se han utilizado para realizar pruebas, entrenar
        modelos, optimizar hiperparámetros y generar gráficas{.}
    \end{minipage}.
}

\vspace{1cm}

\dirtree{%
    .1 /paddel/ \\
    \hphantom{0cm}{}
    \begin{minipage}[t]{10cm}
        \normalfont
        Librería PADDEL, es un proyecto de Python instalable mediante \texttt{pip} que
        contiene el código utilizado para realizar la fase de investigación
        (procesamiento de los vídeos, extracción de características, etc{.}) del que va a
        depender la aplicación web{.}
    \end{minipage}.
    .2 src/paddel/ \\
    \hphantom{0cm}{}
    \begin{minipage}[t]{10cm}
        \normalfont
        Contiene los diferentes archivos de Python que componen la librería PADDEL{.}
    \end{minipage}.
    .3 hyper\_parameters/ \\
    \hphantom{0cm}{}
    \begin{minipage}[t]{10cm}
        \normalfont
        Contiene los archivos de Python relacionados con la fase de optimización de
        hiperparámetros{.}
    \end{minipage}.
    .3 preprocessing/ \\
    \hphantom{0cm}{}
    \begin{minipage}[t]{10cm}
        \normalfont
        Contiene los archivos de Python relacionados con el preprocesado y
        transformación de los vídeos para reducirlos a un conjunto de características{.}
    \end{minipage}.
}

\section{Manual del programador}

En esta sección se detalla todo lo que debería saber una persona que para
realizar cambios sobre las diferentes partes que componen este proyecto.

\subsection{Aplicación web}

Para poder utilizar esta aplicación Docker debe ser instalado con antelación. La
\href{https://docs.docker.com/engine/install/}{documentación de Docker} detalla
el proceso de instalación sobre diferentes plataformas. Una vez Docker está
instalado el proceso siguiente debería ser agnóstico al sistema operativo
utilizado.

La aplicación se compone por un conjunto de contenedores de Docker que
interactúan entre ellos. En la raíz de la aplicación (\texttt{/app/}) se
encuentran varios archivos \textit{docker-compose} de tipo YAML:

\begin{itemize}
    \item \texttt{docker-compose.yml}: Contiene la configuración básica común
          tanto para el entorno de desarrollo como para el de producción.
          Contiene información como dependencias entre contenedores, variables
          de entorno que se pasa a cada contenedor y puertos en los que se va a
          servir la aplicación.
    \item \texttt{docker-compose.dev.yml}: Fichero con la configuración de los
          contenedores específica al entorno de desarrollo. Simplemente
          establece un mapeo entre los directorios del equipo anfitrión y los de
          los contenedores para que los cambios realizados desde el anfitrión se
          vean reflejados dentro de los contenedores y éstos se actualicen de
          forma acorde.
    \item \texttt{docker-compose.prod.yml}: Fichero con la configuración de los
          contenedores específica al entorno de producción. Simplemente
          establece el reinicio automático de los contenedores, para que en caso
          de fallo o reinicio del sistema, los contenedores se lancen junto al
          servicio de Docker.
\end{itemize}

Gracias al uso de Docker Compose se crea una red interna de Docker que conecta
los diferentes contenedores entre sí y establece \textit{hostnames} simples
que pueden utilizar para conectarse unos con otros. Por ejemplo, el proxy
inverso puede realizar una petición HTTP a la API en la ruta
\texttt{http://api}.

La configuración de los distintos contenedores se realiza mediante variables de
entorno. Con el fin de establecer estas variables se ha creado el fichero
\texttt{sample.env} que contiene unos valores básicos para estas variables que
deberán ser cambiadas para adaptarse al entorno en el que se van a ejecutar los
contenedores. Una vez realizados los cambios este fichero debe ser guardado con
el nombre \texttt{.env} en el mismo directorio donde se encuentra
\texttt{sample.env}. Este fichero \texttt{.env} es detectado de forma automática
por Docker cuando se lanzan los contenedores.

Para las operaciones de arranque y parada de los contenedores se utiliza el
comando \texttt{docker compose} junto con los parámetros adecuados para la
operación que se desea realizar. Como estos comando no son triviales y se
utilizan de forma frecuente se encuentran guardados en un fichero
\texttt{Makefile}
. Son posibles los siguientes comandos\footnote{Para utilizar un archivo
    \texttt{Makefile} se necesita la utilidad \texttt{make}, en caso de que no
    se disponga de la misma se pueden copiar los comandos del archivo
    \texttt{Makefile} y utilizar manualmente.}:

\begin{itemize}
    \item \texttt{make down}: Equivalente al comando:
          \begin{flushleft}
              \texttt{docker compose down ----remove-orphans ----rmi all ----volumes ----timeout 0}
          \end{flushleft}
          Elimina los contenedores y todo lo relacionado con los mismos sin
          esperar. No se elimina el caché que guarda Docker, por lo que si se
          vuelve a lanzar los contenedores no es necesario volver a descargar e
          instalar todo.

    \item \texttt{make dev}: Equivalente al comando:
          \begin{flushleft}
              \texttt{make down \&\& \\
                  docker compose -f docker-compose.yml \\
                  -f docker-compose.dev.yml up -d}
          \end{flushleft}
          Para y elimina los contenedores si estos estaban funcionando y los
          lanza en modo desarrollo.

    \item \texttt{make prod}: Equivalente al comando:
          \begin{flushleft}
              \texttt{make down \&\& \\
                  docker compose -f docker-compose.yml \\
                  -f docker-compose.prod.yml up -d}
          \end{flushleft}
          Para y elimina los contenedores si estos estaban funcionando y los
          lanza en modo producción.
\end{itemize}

Para el desarrollo de la aplicación, como es de esperar, se utiliza el comando
\texttt{make dev} para arrancar los contenedores (habiendo antes instalado
Docker y creado el archivo \texttt{.env}).

En modo desarrollo se pueden editar los ficheros directamente desde el sistema
anfitrión y los cambios se van a ver reflejados sobre los contenedores. Esto
puede ser útil para realizar cambios pequeños en los que no se necesiten las
ayudas que puede dar una IDE.

Para utilizar un entorno integrado para realizar el desarrollo se deben utilizar
herramientas que pueden conectarse a contenedores de Docker, como, por ejemplo,
\href{https://code.visualstudio.com/}{Visual Studio Code} con la extensión
\href{https://marketplace.visualstudio.com/items?itemName=ms-vscode-remote.remote-containers}{Dev
Containers} o IDEs de JetBrains, como IntelliJ o Pycharm, con el plugin de
Docker.

En los siguientes apartados se detalla el proceso de desarrollo sobre los
diferentes contenedores.

\subsubsection{Proxy}

El contenedor \textit{proxy} contiene una instancia de Caddy funcionando como
proxy inverso. Caddy es un servidor web similar a Nginx, pero con algunas
características adicionales, como la gestión automática de certificados SSL.

Los archivos que utiliza este contenedor se encuentran en la carpeta
\texttt{/app/proxy}:

\begin{itemize}
    \item \texttt{Dockerfile}: Contiene la imágen y el proceso a seguir que
          Docker debe realizar para construir el contenedor.
    \item \texttt{Caddyfile}: Contiene la configuración de redirecciones de
          Caddy que determina el contenedor de destino para las peticiones que
          recibe.
\end{itemize}

Este contenedor es el único que tiene acceso al exterior, por lo que todas las
peticiones que se hagan a la aplicación van a pasar por él.

\subsubsection{API}

El contenedor \textit{api} se trata de una implementación basada en la librería
FastAPI. Se puede obtener una visión general de las peticiones posibles en la
ruta \texttt{/api} de la localización en la que se encuentra el servicio.

El fichero \texttt{Dockerfile} define como se prepara el contenedor, los pasos
generales que sigue son:

\begin{enumerate}
    \item Copiar (o montar) el código fuente de la API y la librería PADDEL
          desde el anfitrión.
    \item Instalar los requisitos (\texttt{requirements.txt}), que incluyen
          PADDEL y las dependencias específicas de la API.
    \item Se lanza el servicio de FastAPI con los parámetro adecuados
          dependiendo de si se lanza en modo desarrollo o producción.
\end{enumerate}

\subsubsection{Web}

El contenedor \textit{web} es un proyecto basado en NodeJS que utiliza el
framework de JavaScript SvelteKit. Este framework está basado en Svelte, una
librería de JavaScript que permite la creación de componentes interactivos y
reutilizables dentro de una web. Debido a esto, la
\href{https://kit.svelte.dev/docs/introduction}{documentación de SvelteKit} es
un recurso de gran valor para el desarrollo de este proyecto.

En la raíz del proyecto existen varios archivos de interés:

\begin{itemize}
    \item \texttt{Dockerfile}: Define cómo se crea el contendor de Docker.
    \item \texttt{svelte.config.js}: Define algunas configuraciones para
    SvelteKit.
    \item \texttt{package.json}: Especifica detalles sobre el proyecto de NodeJS
          como nombre del proyecto, autor, dependencias y algunos comandos de
          utilidad.
    \item \texttt{package.lock.json}: Este archivo define las versiones
          específicas que se han utilizado de cada dependencia junto con sumas
          de comprobación para asegurar que se puede reproducir el entorno en el
          que se va a ejecutar la aplicación de forma exacta.
    \item \texttt{postcss.config.js}, \texttt{tailwind.config.js}: Archivos
          utilizados para la configuración de Tailwind CSS.
    \item \texttt{vite.config.ts}: Archivo que configura Vite, que es la
    herramienta utilizada para lanzar el servidor local de la página web en el
    entorno de desarrollo.
    \item \texttt{.prettierrc}: Archivo de configuración de Prettier, que es el
    formateador de código utilizado para esta parte de la aplicación.
\end{itemize}

Aunque los archivos anteriores son importantes, la mayoría de cambios y
adiciones se realizan dentro de los directorios \texttt{/src/routes}, que define
la estructura de páginas de la web, y \texttt{/src/lib}, donde se encuentran los
diferentes componentes reutilizables de la web, además de algunos archivos de
TypeScript (\texttt{.ts}) con funciones de utilidad y el archivo \texttt{api.ts}
que implementa una interfaz que refleja las llamadas que se pueden hacer a la
API implementada con FastAPI.

Además, cabe destacar el directorio \texttt{/src/i18n} que contiene los idiomas
soportados por la aplicación en diferentes carpetas, \texttt{en-GB} y
\texttt{es-ES}, con las traducciones. La implementación realizada permite añadir
nuevos idiomas simplemente creando una nueva carpeta con el código de idioma y
país respectivos. Por ejemplo, para añadir el idioma portugues se añadiría la
carpeta \texttt{pt-PT}, se copiarían los archivos de la carpeta \texttt{es-ES} y
editarían para añadir las traducciones necesarias. La página con el nuevo idioma
debería aparecer en el menú de selección de idioma y ser accesible en la ruta
\texttt{/pt-PT/} de la web.

Los archivos estáticos que se utilizan, como imágenes, vídeos, iconos, se
almacenan en el directorio \texttt{/static/}.

En el archivo \texttt{/src/app.css} se definen definen los estilos (CSS)
globales de la web. El contenido de este archivo es bastante escaso debido al
extenso uso de las clases que provee la librería Tailwind CSS.

\subsection{Documentación}

La documentación se trata de un proyecto de \LaTeX{} compuesto por varios
archivos \texttt{.tex} que se compilan a archivos \texttt{.pdf}.

Para poder compilar estos archivos se necesita tener instalada una distribución
de \LaTeX{}, en general la más recomendada es
\href{https://www.tug.org/texlive/}{TeX Live}, disponible en varias plataformas.

Una vez instalado \LaTeX{} se debería tener acceso al comando \texttt{latexmk},
que se utiliza dentro del archivo \texttt{Makefile} tanto para compilar los
archivos como para limpiar los archivos sobrantes. Están definidos los
siguientes comandos:

\begin{itemize}
    \item \texttt{make memoria}: Compila la memoria. Equivalente a ejecutar:
          \begin{flushleft}
              \texttt{latexmk -cd -pdf memoria.tex}
          \end{flushleft}
    \item \texttt{make anexos}: Compila los anexos. Equivalente a ejecutar:
          \begin{flushleft}
              \texttt{latexmk -cd -pdf anexos.tex}
          \end{flushleft}
    \item \texttt{make all}: Compila la memoria y los anexos. Equivalente a
          ejecutar los comandos anteriores.
    \item \texttt{make clean}: Elimina archivos auxiliares generados durante la
          compilación. Equivalente a ejecutar:
          \begin{flushleft}
              \texttt{latexmk -cd -pdf -bibtex-cond1 -c memoria.tex} \\
              \texttt{latexmk -cd -pdf -bibtex-cond1 -c anexos.tex}
          \end{flushleft}
\end{itemize}

Para editar los archivos de este proyecto existen algunas alternativas, como el
entorno integrado creado específicamente para \LaTeX{}
\href{https://www.xm1math.net/texmaker/}{Texmaker} o Visual Studio Code con la
extensión
\href{https://marketplace.visualstudio.com/items?itemName=James-Yu.latex-workshop}{Latex
    Workshop}.

\subsection{Librería PADDEL}

La libraría PADDEL es una librería de Python que se puede instalar mediante el
comando \texttt{pip} pasándole la ruta donde se encuentra el archivo
\texttt{pyproject.toml}, por ejemplo:

\texttt{pip install ./paddel}

La librería contiene diferentes módulos que se utilizan para el preprocesado de
los vídeos del paciente y el entrenamiento de modelos.

\subsubsection{Configuración}

Para configurar algunos parámetros del funcionamiento se ha optado por utilizar
variables de entorno. Gracias a la librería
\href{https://docs.pydantic.dev/latest/}{Pydantic} las siguientes variables de
entorno son leídas e interpretadas cuando el módulo sea importado por primera
vez:

\begin{itemize}
    \item \texttt{PADDEL\_MIN\_DETECTION\_SECONDS}: Tiempo mínimo de detección en
    segundos para que un vídeo no se descarte debido a no tener suficientes
    fotogramas seguidos con poses detectadas. Por defecto es 15 segundos.
    \item \texttt{PADDEL\_ROLLING\_STD\_SECONDS}: Ventana utilizada para el cálculo
    de la desviación estándar móvil en la extracción de algunas características,
    esto se utiliza para obtener un umbral y, así, saber cuando considerar o no
    máximos locales dentro de una secuencia de valores. Por defecto es 3
    segundos.
    \item \texttt{PADDEL\_SLOT\_SIZE\_SECONDS}: Para las características que se
    calculan como una diferencia entre la parte inicial y final de una secuencia
    de poses (vídeo), es el tamaño de cada una de estas partes. No puede ser
    mayor que \texttt{PADDEL\_MIN\_DETECTION\_SECONDS} y establecerlo a más de
    $\frac{1}{2}$ de su valor puede ser problemático. Por defecto es 7 segundos.
    \item \texttt{PADDEL\_MAX\_PROCESSES}: Número máximo de procesos a utilizar en
    operaciones paralelizadas. Por defecto es el número de núcleos del
    procesador de la máquina donde se ejecute la librería.
\end{itemize}

\subsubsection{Tipos}

Esta librería utiliza extensamente el tipado estático que ofrece Python por
defecto. Aunque este no es impuesto en tiempo de ejecución, resulta muy útil al
ser utilizado con un entorno de desarrollo integrado (IDE) ya que se pueden
obtener ayudas y avisos que no se podría de otro modo.

El archivo \texttt{types.py} contine las definiciones de algunos tipos complejos
que se utilizan a través de la librería debido a que son específicos a este
problema concreto. Algunos de estos tipos incluyen:

\begin{itemize}
    \item \texttt{Image}: Definido como un array de Numpy de cualquier dimensión
    que tenga como tipo de dato números enteros sin signo de 8 bits, lo que se
    corresponde con imágenes en RGB, RGBA, BGR, etc.
    \item \texttt{Video}: Definido como un objeto iterable de objetos de tipo
    \texttt{Image}, lo que es lógico, ya que un vídeo es una secuencia de
    imágenes.
    \item \texttt{Point}: Definido como una tupla con componenetes \textit{x},
    \textit{y} y \textit{z} representa un punto en un espacio de tres
    dimensiones, es utilizado para definir las \textit{landmarks} que componen
    una pose de la mano.
\end{itemize}

\subsubsection{Convenciones del código}

En la librería se han seguido las convenciones de formato definidas por
\href{https://github.com/psf/black}{Black}, que es una libería de formateado de
código para Python, escrita en Python.

Se ha seleccionado Black debido a que incluye muchas opiniones predefinidas y
limita en gran medida la configuración que puede controlar el usuario. Esto
puede parecer una desventaja, pero en este caso simplemente se deseaba una
utilidad que diese un aspecto homogéneo al código con el menor esfuerzo posible
por parte del programador. Además, gracias a la limitada personalización que
ofrece Black, se cumple con el principio de Convención sobre Configuración
\cite{eswiki:125269135}.

Además de Black, se utilizan las herramientas
\href{https://github.com/PyCQA/autoflake}{autoflake},
\href{https://github.com/PyCQA/isort}{isort} y
\href{https://github.com/python/mypy}{mypy} para eliminar sentencias
\texttt{import} y variables no utilizadas, reordenar las sentencias
\texttt{import} y comprobar el tipado estático utilizado respectivamente.

Como antes, el uso de estas herramientas se simplifica gracias al uso de la
utilidad \texttt{make}. En este caso el archivo \texttt{Makefile} aporta los
siguientes comandos:

\begin{itemize}
    \item \texttt{make lint}: Realiza un análisis del código con las
    herramientas anteriores notificando de posibles problemas pero sin realizar
    correcciones.
    \item \texttt{make format}: Es igual que el comando anterior, pero se
    aplican correcciones al código.
\end{itemize}

\subsection{Notebooks de Jupyter}

Para ejecutar los notebooks de Jupyter del proyecto se debe instalar la librería
Paddel, además de Jupyter en el sistema anfitrión. Para esto se puede utilizar
el archivo \texttt{requirements.txt} presente en la raíz del proyecto:

\texttt{pip install -r requirements.txt}

Tras el paso anterior se puede lanzar el kernel de IPython (backend que utiliza
Jupyter) mediante el comando:

\texttt{jupyter notebook}

Este comando inicia un servidor web al que se puede acceder de forma local
mediante cualquiera de los enlaces que se muentran por consola (acceder de forma
remota resulta algo más complicado dependiendo de casos). En esta página web se
muestra un explorador de archivos mediante el que se puede navegar a la
localización donde se encuentran los notebooks de Jupyter (\texttt{/notebooks/})
y abrir el notebook que se desee utilizar.

Existen los siguientes notebooks:

\begin{itemize}
    \item \texttt{00-train.ipynb}: Utilizado para entrenar una multitud de
    modelos, buscar los parámetros óptimos de los mismos mediante la técnica
    conocida como \textit{grid search} y exportar los resultados a un fichero
    \texttt{.csv}.
    \item \texttt{01-plots.ipynb}: Utiliza los resultados obtenidos en la
    ejecución del notebook anterior para crear diferentes gráficas que pueden
    revelar información adicional sobre el comportamiento de los algoritmos
    utilizados con diferentes combinaciones de parámetros y conjuntos de datos
    de entrada.
    \item \texttt{02-model.ipynb}: Utilizado para entrenar un modelo final que
    vaya a ser utilizado en la aplicación web para realizar las predicciones. En
    este caso el entrenamiento se realiza con el conjunto de datos completo. Los
    parametros utilizados para el modelo serán los que se consideren más
    adecuados en base al los resultados observados en la ejecución de los
    notebooks anteriores.
\end{itemize}

\section{Compilación, instalación y ejecución del proyecto}

La mayor parte de este proyecto ha sido realizada Python, un lenguaje
interpretado (no compilado), por lo que estas partes no necesitan ser
compiladas.

\subsection{API}

Para lanzar la API se necesita un entorno con la versión 3.9 de Python
disponible. Dependiendo del sistema operativo puede ser necesario instalar la
librería \href{https://ffmpeg.org}{FFmpeg} para poder realizar el procesado de
vídeo al realizar las predicciones.

A continuación se deberán instalar las dependencias recogidas en el fichero
\texttt{requirements.txt} mediante:

\texttt{pip install -r requirements.txt}

Además, la API necesita tener acceso a una base de datos PostgreSQL para
almacenar de forma persistenete la información con la que trabaja. Se puede
comunicar la información de acceso a dicha base de datos mediante las siguientes
variables de entorno:

\begin{itemize}
    \item \texttt{DB\_USER}: Usuario a utilizar para acceder a la base de datos.
    \item \texttt{DB\_PASSWORD}: Constraseña para el usuario definido por
    \texttt{DB\_USER}.
    \item \texttt{DB\_NAME}: Nombre de la base de datos a utilizar.
\end{itemize}

Para controlar la sesión de los usuarios se utilizan cadenas JWT (JSON Web
Token), las cuales deben ser firmadas mediante una clave secreta que también
debe ser establecida mediante una variable de entorno denominada
\texttt{SECRET\_KEY}. Esta variable puede ser cualquier cadena de texto pero es
recomendable que sea generada mediante una herramienta de criptografía diseñada
para este propósito, como el comando \texttt{openssl} presente en la mayoría de
distribuciones de Linux:

\texttt{openssl rand -hex 32}

Con el que se obtiene una salida similar a la siguiente:

\texttt{5e95e9e53d95d2892a63682a118a834429f43f3fd3eb951eda781-}\\
\texttt{1d9a1ad3f64}

Una vez las dependencias están instaladas y las variables de entorno
establecidas se puede lanzar el servidor de la API de forma local mediante:

\texttt{uvicorn app.main:app}

Gracias a utilizar FastAPI se crea automáticamente la ruta \texttt{/docs} que
presenta una interfaz que muestra las diferentes acciones que permite hacer la
API además de permitir ejecuar estas acciones para comprobar su correcto
funcionamiento de forma bastante conveniente.

\subsection{Web}

Para utilizar la aplicación web se necesita un entorno que disponga de
\href{https://nodejs.org/}{NodeJS}, en concreto la versión 18 es la ideal, al
ser la utilizada durante el desarrollo del proyecto. Además, si no se ha
instalado con NodeJS, se debe instalar \href{https://www.npmjs.com/}{NPM}
también, que es el gestor de paquetes de Node.

Para instalar las dependencias del proyecto se utiliza:

\texttt{npm install} o \texttt{npm i}

Los comandos anteriores leen los archivos \texttt{package.json} y \linebreak
\texttt{package-lock.json} e instalan los paquetes declarados en estos.

La aplicación web ha sido realizada mediante el framework SvelteKit, y utiliza
archivos \texttt{.ts} y \texttt{.svelte} que no son interpretables por un
navegador, por lo que deben ser convertidos a únicamente ficheros
\texttt{.html}, \texttt{.css} y \texttt{.js}. Para esto se realiza una
compilación mediante el comando \texttt{npm run build} que realiza la conversión
anterior además de realizar ciertas optimizaciones como eliminar código no
utilizado para reducir el tamaño final de los archivos. Una vez realizada la
compilación, los archivos finales se almacenan en el directorio
\texttt{/build/}.

Esta aplicación web sirve como \textit{frontend} para la API, por lo que
necesita saber cómo comunicarse con la ella, para esto se utiliza la variable de
entorno \texttt{PUBLIC\_API\_LOCATION} que podría tomar el valor
\texttt{http://localhost:44444} por ejemplo.

\subsection{Ejecución mediante Docker}

Este proyecto está compuesto por varias piezas que interactúan entre sí.
Instalar todos los programas y dependencias, además de desplegar una base de
datos, puede resultar una tarea ardua y tediosa. Asimismo, dependiendo del
sistema sobre el que se esté intentando realizar el despliegue, pueden existir
inconsistencias en los procesos definidos anteriormente.

Debido a lo anterior se ha optado por utilizar la conocida herramienta
\href{https://www.docker.com/}{Docker}, gracias a esto se pueden definir
diferentes contenedores con entornos controlados sobre los que se pueden
realizar los pasos de instalación de forma determinista (obteniendo siempre los
mismos resultados).

El único requisito para realizar un despliegue mediante Docker es, valga la
redundancia, tener Docker instalado en el sistema.

Para establecer las diferentes variables de entorno mencionadas anteriormente se
utiliza un fichero de variables de entorno denominado \texttt{.env}, un ejemplo
con los posibles valores de estas variables se encuentra en el fichero
\texttt{sample.env}.

Para realizar el despliegue se utiliza el comando \texttt{make prod} desde la
ruta \texttt{/app/}, donde sen encuentran los archivos
\texttt{docker-compose.yml}. Si no se dispone de la utilidad \texttt{make} se
puede utilizar:

\texttt{docker compose -f docker-compose.yml -f docker-compose.prod.yml up -d}

Este comando realiza varias tareas:

\begin{itemize}
    \item Inicia una base de datos PostgreSQL.
    \item Lanza el servicio de la API.
    \item Compila e inicia el servicio de la web.
    \item Lanza un proxy inverso para permitir el acceso a la API y a la web
    mediante distintos subdominios.
\end{itemize}

Una vez lanzados los contenedores se pueden usar las urls establecidas en el
fichero \texttt{.env} para acceder a los servicios, por defecto:
\texttt{https://localhost} para acceder a la web y
\texttt{https://api.localhost} para acceder a la API.

\section{Pruebas del sistema}

\newcounter{ncp}
\setcounter{ncp}{0}

\newcolumntype{L}[1]{>{\hsize=#1\hsize\RaggedRight} X}

\newcommand{\cp}[7]{
    \addtocounter{ncp}{1}

    \raggedright
    \begin{table}[H]
    \begin{tabular}{p{0.3\linewidth}p{0.6\linewidth}}
    
    \toprule
    Identificador & CP\decimal{ncp} \\
    Prioridad & #1 \\
    Fecha de ejecución & #2 \\
    Objetivo & #3 \\
    Precondiciones & {#5} \\
    Postcondiciones & {#6} \\
    \specialrule{.8pt}{2pt}{0pt}

    \multicolumn{2}{@{}p{\textwidth}@{}}{
        \begin{hyphenrules}{nohyphenation}
        \bgroup
        \rowcolors{2}{gray!20}{white}
        \def\arraystretch{1}
        \setlength\tabcolsep{2px}
        \scriptsize
        \begin{tabularx}{\textwidth}{L{0.2}L{0.25}L{0.25}L{0.25}>{\centering\arraybackslash}L{0.05}}
            \rowcolor[gray]{.9}
            \textbf{Acción} & \textbf{Entrada} & \textbf{Salida Esperada} & \textbf{Salida Real} & \\
            \specialrule{.4pt}{0pt}{2pt}
            #7
        \end{tabularx}
        \egroup
        \end{hyphenrules}
    } \\

    \specialrule{.8pt}{0pt}{0pt}
    \end{tabular}
    \caption{CP\decimal{ncp}: #4}
    \end{table}
}

\cp {Alta}
    {08/06/2023}
    {Probar la obtención de una predicción}
    {Predicción}
    {-}
    {-}
    {
        Acceder a la página del formulario de predicción &  & Estar en la página del formulario & Estar en la página del formulario & \ding{51} \\
        Introducir información & Mano dominante: Derecha; Mano del vídeo: Derecha; Edad: 71; Sexo: Femenino; Archivo de vídeo: archivo respectivo; & Formulario relleno & Formulario relleno & \ding{51} \\
        Enviar información &  & Aparece pantalla de carga & Aparece pantalla de carga & \ding{51} \\
        Esperar a que finalice el procesado &  & Probabilidad <50\% de padecer Parkinson & Probabilidad de 7.13\% de padecer Parkinson & \ding{51} \\
    }

\cp {Alta}
    {08/06/2023}
    {No tener una sesión iniciada}
    {}
    {El usuario se encuentra en la página de inicio de sesión}
    {El usuario ha accede al panel de administrador}
    {
         &  &  &  & \ding{51} \\
    }

\cp {Alta}
    {08/06/2023}
    {Probar el cierre de sesión}
    {Cierre de Sesión}
    {El usuario se encuentra en el panel de administrador}
    {El usuario ha finalizado su sesión}
    {
         &  &  &  & \ding{51} \\
    }

\cp {Alta}
    {08/06/2023}
    {Probar añadir modelo}
    {Añadir modelo}
    {El usuario se encuentra en el panel de administración de modelos}
    {Un modelo ha sido añadido}
    {
         &  &  &  & \ding{51} \\
    }

\cp {Alta}
    {08/06/2023}
    {Probar selección de modelo}
    {Seleccionar modelo}
    {El usuario se encuentra en el panel de administración de modelos}
    {Un modelo diferente ha sido seleccionado}
    {
         &  &  &  & \ding{51} \\
    }

\cp {Alta}
    {08/06/2023}
    {Probar eliminar modelo}
    {Eliminar modelo}
    {El usuario se encuentra en el panel de administración de modelos}
    {Un modelo diferente ha sido seleccionado}
    {
         &  &  &  & \ding{51} \\
    }

\cp {Alta}
    {08/06/2023}
    {Probar añadir usuario}
    {Añadir usuario}
    {El usuario se encuentra en el panel de administración de usuarios}
    {Un usuario ha sido añadido}
    {
         &  &  &  & \ding{51} \\
    }

\cp {Alta}
    {08/06/2023}
    {Probar eliminar usuario}
    {Eliminar usuario}
    {El usuario se encuentra en el panel de administración de usuarios}
    {Un modelo diferente ha sido seleccionado}
    {
         &  &  &  & \ding{51} \\
    }

\cp {Alta}
    {08/06/2023}
    {Probar cambio de idioma}
    {Cambiar idioma}
    {El usuario se encuentra en cualquier página de la aplicación}
    {El idioma de las cadenas de texto que se muetran ha cambiado}
    {
         &  &  &  & \ding{51} \\
    }
