\capitulo{Especificación de diseño}
\label{cha:Especificación de diseño}

\section{Introducción}

La especificación de diseño sirve como una guía para el proceso de diseño de la
aplicación, de modo que todas las personas involucradas en el proyecto saben
cómo debería ser el producto final y pueden comunicarse y colaborar de forma más
fácil.

En este anexo se intenta definir de forma clara el modo en que se van a
almacenar los datos, las distintas interacciones que deberán existir entre las
partes de la aplicación y la forma en la que el usuario final deberá interactuar
con ellas.

\section{Diseño de datos}

En este apartado se explica de forma detallada la forma en la que se almacenan
los datos con los que trabaja la aplicación (información de usuario, información
sobre los modelos y los archivos binarios de estos modelos).

En este caso se ha optado por utilizar el sistema gestor de bases de datos
PostgreSQL. Esta decisión es debido a que ya existía cierta familiaridad con el
programa y a que es ampliamente soportado por SQLAlchemy, la librería ORM
(Object-Relational Mapping) utilizada en la API.

El diseño de datos creado es muy simple debido a que la aplicación en sí es muy
simple. Su objetivo principal es permitir la fácil utilización de modelos
previamente entrenados para obtener predicciones y selección de los mismos por
parte de los administradores.

En la base de datos se almacenan las siguientes entidades:

\begin{itemize}
    \item \textbf{Usuario (User)}: Corresponde con los usuarios registrados en
    la aplicación. Dispone de los campos \textit{id}, \textit{username} y
    \textit{password}.
    \item \textbf{Modelo (Model)}: Corresponde con los modelos disponibles
    dentro de la aplicación. Dispone de los campos \textit{id}, \textit{name},
    \textit{path} y \textit{selected}.
\end{itemize}

Para determinar el modelo seleccionado se utiliza el campo \textit{selected},
que es de tipo booleano, puede valer \textit{null} y su valor es único. De modo
que este campo vale \textit{true} para el modelo seleccionado y \textit{null}
para todos los demás.

\section{Diseño procedimental}

\section{Diseño arquitectónico}

El diseño arquitectónico de la aplicación fue seleccionado sin consultar ningún
patrón preexistente. Símplemente se tomaron las decisiones que parecían más
lógicas en su momento. Pese a esto, la aplicación ha tomado una estructura muy
similar al patrón arquitectónico conocido como MVP
(\textit{Model-View-Presenter})
