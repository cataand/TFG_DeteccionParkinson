\capitulo{Plan de Proyecto Software}
\label{cha:Plan de Proyecto Software}

\section{Introducción}

Este anexo detalla el procedimiento que se ha seguido para gestionar el progreso
del proyecto y la realización de tareas.

\section{Planificación temporal}

La planificación del proyecto se llevará a cabo mediante una metodología ágil,
en concreto Scrum, con el apoyo de la herramienta
\href{https://www.zenhub.com/}{ZenHub} para la gestión de tareas y sprints
dentro de GitHub. En este caso cada sprint tiene una duración de una semana.

\subsection{Preparación}

Antes de comenzar el sprint 1 se realizaron las siguientes tareas:

\begin{itemize}
    \item Organizar el proyecto en diferentes carpetas con las partes que lo
          componen.
    \item Implementar la conectividad básica entre los contenedores de Docker
          que implementan la aplicación web.
    \item Implementar el parseado de los nombres de archivo de los vídeos.
    \item Implementar la extracción de puntos del esqueleto de la mano mediante
          Mediapipe Hands.
    \item Crear un paquete instalable o librería de Python con las utilidades
          que se utilizan en la fase de investigación para facilitar su uso
          posterior en la aplicación web.
\end{itemize}

Haber decidido en esta fase inicial la estructura que va a tener el proyecto y
de forma aproximada las herramientas que van a ser utilizadas a ayudado en gran
medida durante el desarrollo, debido a que las funcionalidades de todas las
piezas están bien definidas y separadas.

\subsection{Sprint 1 (6/2/2023 -- 12/2/2023)}

En este sprint se avanzó en la parte del preprocesado de los vídeos, extrayendo
diferentes atributos a partir del esqueleto que obtenido a través de MediaPipe.

Además, se separó el proceso de construcción de los contenedores de Docker
dependiento de si se lanzan en un entorno de desarrollo o en producción.
Optimizando el despliegue en producción, mientras que, en desarrollo, se
muestran mensajes adicionales y observa los archivos para reaccionar a cambios.

Tareas realizadas:

\begin{itemize}
    \item Añadir extracción de frecuencia de toques a partir de la secuencia de
          poses de la mano extraida por Mediapipe.
    \item Añadir extracción de diferencia entre la frecuencia de toques del
          intervalo de tiempo inicial y final de la secuencia de poses.
    \item Establecer ángulo máximo para la detección de toques.
    \item Cambiar la configuración de los contenedores de Docker para usar
          multietapa, separando la configuración para los entornos de desarrollo
          y producción.
\end{itemize}

\subsection{Sprint 2 (13/2/2023 -- 19/2/2023)}

Este sprint se centra en el preprocesado de los vídeos, añadiendo la extracción
de varias características utilizadas en trabajos anteriores en este área de las
series temporales con las poses de la mano.

Tareas realizadas:

\begin{itemize}
    \item Añadir extracción de amplitud media.
    \item Añadir extracción de diferencia de amplitud media entre intervalo
          inicial y final del vídeo.
    \item Añadir extracción de variación de amplitud.
    \item Añadir extracción de velocidad del movimiento.
    \item Comenzar manual del programador.
    \item Reemplazar Nginx por Caddy como proxy inverso de la aplicación.
    \item Actualizar fichero README.md.
\end{itemize}

\subsection{Sprint 3 (20/2/2023 -- 26/2/2023)}

En este sprint se añadió la extracción de caracerísticas mediante la librería
TSFresh que extrae al rededor de 750 características adicionales a las que se
extraían hasta este punto.

Tareas realizadas:

\begin{itemize}
    \item Añadir extracción de características mediante TSFresh.
    \item Cambiar secuencia de instalación del contenedor de Docker para la API
          mejorando el cacheado de los pasos.
\end{itemize}

\subsection{Sprint 4 (27/2/2023 -- 5/3/2023)}

En este sprint se cambió la extracción de características propias para
aprovechar las metodologías ya implementadas por TSFresh, además de realizar
algunos cambios sobre la memoria.

Tareas realizadas:

\begin{itemize}
    \item Añadir herramientas de desarrollo a la memoria.
    \item Cambiar implementación de extracción de características de trabajos
          previos para que sea compatible con la librería TSFresh.
\end{itemize}

\subsection{Sprint 5 (6/3/2023 -- 12/3/2023)}

Tareas realizadas:

\begin{itemize}
    \item Actualizar sección de herramientas de desarrollo.
\end{itemize}

\subsection{Sprint 6 (13/3/2023 -- 19/3/2023)}

En este sprint se comienzan a entrenar los primeros modelos de aprendizaje
mediante una técnica que se conoce como \textit{grid search}. Además se mejora
la implementación de la extracción de las marcas temporales de las series
temporales.

Tareas realizadas:

\begin{itemize}
    \item Implementar optimización de hiperparámetros para varios modelos
          mediante \textit{grid search}.
    \item Añadir información temporal a la extracción de poses de Mediapipe a
          partir de la tasa de fotogramas.
\end{itemize}

\subsection{Sprint 7 (20/3/2023 -- 26/3/2023)}

En este sprint se cambian algunos detalles sobre la optimización de
hiperparámetros, utilizando más modelos y cambiando los parámetros de la
validación cruzada.

Tareas realizadas:

\begin{itemize}
    \item Añadir Perceptrón multicapa, Adaboost y XGBoost a los modelos de la
          optimización de hiperparámetros.
    \item Cambiar validación cruzada para utilizar 5 repeticiones de 2
          \textit{folds}.
    \item Sustituir características de mano grabada y mano dominante por una
          única características, si está utilizando la mano dominante.
\end{itemize}

\subsection{Sprint 8 (27/3/2023 -- 2/4/2023)}

En este sprint se crean visualizaciones de los resultados obtenidos durante la
optimización de hiperparámetros, además, se añade un parámetro adiciona, el
número de características a seleccionar.

Tareas realizadas:

\begin{itemize}
    \item Arreglar Makefile para la compilación de la documentación.
    \item Añadir generación de gráficas con los resultados obtenidos de la
    optimización de hiperparámetros mediante \textit{grid search}.
    \item Añadir selección del número de características a utilizar a la
          optimización de hiperparámetros.
    \item Refactorizar librería.
    \item Cambiar el framework de JavaScript de la web de SvelteKit a NextJS.
    \item Implementar una barra de navegación básica.
\end{itemize}

\subsection{Sprint 9 (3/4/2023 -- 9/4/2023)}

En este sprint se realizaron varios cambios sobre los diferentes componentes del
proyecto.

Tareas realizadas:

\begin{itemize}
    \item Actualizar el manual del programador con los cambios realizados hasta
    este punto.
    \item Añadir requisitos de la apliación web.
    \item Añadir diagrama general de casos de uso.
    \item Cambiar la estructura de directorios del frontend de la web.
    \item Añadir y configurar contenedor de base de datos PostgreSQL al proyecto
    de Docker.
    \item Configurar la API para interactuar con la base de datos, creando, los
    modelos de SQLAlchemy con los que se va a trabajar.
    \item Añadir algunos endpoints para permitir una gestión muy básica de
    usuarios.
    \item Bloquear la versión de los contenedores de Docker para evitar
    problemas de dependencias en el futuro.
    \item Añadir persistencia de datos entre creaciones del contenedor de la
    base de datos.
    \item Bloquear versiones de las dependencias de Python. Debido a la versión
    2 de Pandas, se rompe parte de la funcionalidad.
    \item Implementar una versión inicial del endpoint para obtener una
    predicción a partir de un vídeo.
\end{itemize}

\subsection{Sprint 10 (10/4/2023 -- 16/4/2023)}

En este sprint se realizaron varias tareas sobre la aplicación web para permitir
la fácil interacción con el formulario para obtener predicciones.

Tareas realizadas:

\begin{itemize}
    \item Añadir y estilizar un formulario básico para interactuar desde el frontend con la
    API y obtener un predicción.
    \item Utilizar JavaScript para enviar los contenidos de este formulario de
    forma correcta al endpoint.
    \item Añadir barra de progreso de subida del vídeo una vez el formulario es enviado.
    \item Implementar el entrenamiento del modelo final con todos los datos y
    añadirlo a la API para poder realizar la predicción.
    \item Añadir funcionalidad de arrastrar y soltar al formulario para subir el
    vídeo.
\end{itemize}

\subsection{Sprint 11 (17/4/2023 -- 23/4/2023)}

En este sprint se internacionalizó la aplicación web, entre otras cosas de menor
importancia.

Tareas realizadas:

\begin{itemize}
    \item Crear prototipos (mockups) para la página principal y para la gestión de modelos.
    \item Internacionalizar la aplicación mediante TypeSafe-i18n y añadir un selector de idioma.
    \item Añadir autenticación mediante tókenes JWT (JSON Web Token) a la API.
\end{itemize}

\subsection{Sprint 12 (24/4/2023 -- 30/4/2023)}

En este sprint se solucionaron algunos bugs y mejoró la accesibilidad de la
apliación web en gran medida gracias a un
\href{https://www.udacity.com/course/web-accessibility--ud891}{curso sobre
accesibilidad} que se realizó.

Tareas realizadas:

\begin{itemize}
    \item Implementar la autenticación de usuario con la API desde la página web.
    \item Refactorizar y limpiar el código de la web.
    \item Solucionar bug de idioma indefinido.
    \item Pulir los estilos de algunos elementos de la web.
    \item Solucionar bug pantalla de administración se muestra brevemente antes de login.
    \item Mejorar accesibilidad del HTML.
    \item Añadir mensajes acompañando a la predicción dependiendo del resultado.
\end{itemize}

\subsection{Sprint 13 (1/5/2023 -- 7/5/2023)}

En este sprint se implementó la administración de usuarios desde la página web.

\begin{itemize}
    \item Pulir la pantalla de resultados de la web.
    \item Añadir ejecución de un script SQL por defecto en la creación de la
    base de datos.
    \item Añadir endpoints para crear y eliminar usuarios en la API.
    \item Añadir página de gestión de usuarios.
\end{itemize}

\section{Estudio de viabilidad}

\subsection{Viabilidad económica}

\subsection{Viabilidad legal}
