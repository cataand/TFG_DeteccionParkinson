\capitulo{Plan de Proyecto Software}
\label{cha:Plan de Proyecto Software}

\section{Introducción}

El propósito del plan de proyecto es definir de forma clara cuales son los
objetivos del proyecto y de qué manera van a ser alcanzados.

\section{Planificación temporal}

La planificación del proyecto se llevó a cabo mediante una metodología ágil, en
concreto Scrum, en un principio se utilizó la herramienta
\href{https://www.zenhub.com/}{ZenHub}, pero debido a cambios en la política de
la misma, dejó de ser utilizable en este proyecto.

Scrum tiene por objetivo llevar a cabo una progresión incremental, dividiendo el
tiempo de trabajo en partes denominadas \textit{sprint}. Antes de comenzar cada
uno de estos \textit{sprints} se realiza una reunión de planificación con los
miembros del equipo en la que se asigna un determinado número de tareas que se
intentará realizar durante dicho \textit{sprint}. Las tareas realizadas durante
el periodo de tiempo asignado al \textit{sprint} se marcan como completadas, las
que no puedan ser realizadas a tiempo pasarán al siguiente \textit{sprint}.

En este caso concreto la duración de cada \textit{sprint} se ha fijado a una
semana.

\subsection{Preparación}

Antes de comenzar el sprint 1 se realizaron las siguientes tareas:

\begin{itemize}
    \item Organizar el proyecto en diferentes carpetas con las partes que lo
          componen.
    \item Implementar la conectividad básica entre los contenedores de Docker
          que implementan la aplicación web.
    \item Implementar el parseado de los nombres de archivo de los vídeos.
    \item Implementar la extracción de puntos del esqueleto de la mano mediante
          Mediapipe Hands.
    \item Crear un paquete instalable o librería de Python con las utilidades
          que se utilizan en la fase de investigación para facilitar su uso
          posterior en la aplicación web.
\end{itemize}

Haber decidido en esta fase inicial la estructura que va a tener el proyecto y,
de forma aproximada, las herramientas que van a ser utilizadas a ayudado en gran
medida durante el desarrollo, debido a que las funcionalidades de todas las
piezas que componen el proyecto están bien definidas y separadas.

\subsection{Sprint 1 (6/2/2023 -- 12/2/2023)}

En este sprint se avanzó en la parte del preprocesado de los vídeos, extrayendo
diferentes atributos a partir del esqueleto que obtenido a través de MediaPipe.

Además, se separó el proceso de construcción de los contenedores de Docker
dependiento de si se lanzan en un entorno de desarrollo o en producción.
Optimizando el despliegue en producción, mientras que, en desarrollo, se
muestran mensajes adicionales y observa los archivos para reaccionar a cambios.

Tareas realizadas:

\begin{itemize}
    \item Añadir extracción de frecuencia de toques a partir de la secuencia de
          poses de la mano extraida por Mediapipe.
    \item Añadir extracción de diferencia entre la frecuencia de toques del
          intervalo de tiempo inicial y final de la secuencia de poses.
    \item Establecer ángulo máximo para la detección de toques.
    \item Cambiar la configuración de los contenedores de Docker para usar
          multietapa, separando la configuración para los entornos de desarrollo
          y producción.
\end{itemize}

\subsection{Sprint 2 (13/2/2023 -- 19/2/2023)}

Este sprint se centra en el preprocesado de los vídeos, añadiendo la extracción
de varias características utilizadas en trabajos anteriores en este área de las
series temporales con las poses de la mano.

Tareas realizadas:

\begin{itemize}
    \item Añadir extracción de amplitud media.
    \item Añadir extracción de diferencia de amplitud media entre intervalo
          inicial y final del vídeo.
    \item Añadir extracción de variación de amplitud.
    \item Añadir extracción de velocidad del movimiento.
    \item Comenzar manual del programador.
    \item Reemplazar Nginx por Caddy como proxy inverso de la aplicación.
    \item Actualizar fichero README.md.
\end{itemize}

\subsection{Sprint 3 (20/2/2023 -- 26/2/2023)}

En este sprint se añadió la extracción de caracerísticas mediante la librería
TSFresh que extrae alrededor de 750 características adicionales a las que se
extraían hasta este punto.

Tareas realizadas:

\begin{itemize}
    \item Añadir extracción de características mediante TSFresh.
    \item Cambiar secuencia de instalación del contenedor de Docker para la API
          mejorando el cacheado de los pasos.
\end{itemize}

\subsection{Sprint 4 (27/2/2023 -- 5/3/2023)}

En este sprint se cambió la extracción de características propias para
aprovechar las metodologías ya implementadas por TSFresh, además de realizar
algunos cambios sobre la memoria.

Tareas realizadas:

\begin{itemize}
    \item Añadir herramientas de desarrollo a la memoria.
    \item Cambiar implementación de extracción de características de trabajos
          previos para que sea compatible con la librería TSFresh.
\end{itemize}

\subsection{Sprint 5 (6/3/2023 -- 12/3/2023)}

Tareas realizadas:

\begin{itemize}
    \item Actualizar sección de herramientas de desarrollo.
\end{itemize}

\subsection{Sprint 6 (13/3/2023 -- 19/3/2023)}

En este sprint se comienzan a entrenar los primeros modelos de aprendizaje
mediante una técnica que se conoce como \textit{grid search}. Además se mejora
la implementación de la extracción de las marcas temporales de las series
temporales.

Tareas realizadas:

\begin{itemize}
    \item Implementar optimización de hiperparámetros para varios modelos
          mediante \textit{grid search}.
    \item Añadir información temporal a la extracción de poses de Mediapipe a
          partir de la tasa de fotogramas.
\end{itemize}

\subsection{Sprint 7 (20/3/2023 -- 26/3/2023)}

En este sprint se cambian algunos detalles sobre la optimización de
hiperparámetros, utilizando más modelos y cambiando los parámetros de la
validación cruzada.

Tareas realizadas:

\begin{itemize}
    \item Añadir Perceptrón multicapa, Adaboost y XGBoost a los modelos de la
          optimización de hiperparámetros.
    \item Cambiar validación cruzada para utilizar 5 repeticiones de 2
          \textit{folds}.
    \item Sustituir características de mano grabada y mano dominante por una
          única características, si está utilizando la mano dominante.
\end{itemize}

\subsection{Sprint 8 (27/3/2023 -- 2/4/2023)}

En este sprint se crean visualizaciones de los resultados obtenidos durante la
optimización de hiperparámetros, además, se añade un parámetro adiciona, el
número de características a seleccionar.

Tareas realizadas:

\begin{itemize}
    \item Arreglar Makefile para la compilación de la documentación.
    \item Añadir generación de gráficas con los resultados obtenidos de la
    optimización de hiperparámetros mediante \textit{grid search}.
    \item Añadir selección del número de características a utilizar a la
          optimización de hiperparámetros.
    \item Refactorizar librería.
    \item Cambiar el framework de JavaScript de la web de SvelteKit a NextJS.
    \item Implementar una barra de navegación básica.
\end{itemize}

\subsection{Sprint 9 (3/4/2023 -- 9/4/2023)}

En este sprint se realizaron varios cambios sobre los diferentes componentes del
proyecto.

Tareas realizadas:

\begin{itemize}
    \item Actualizar el manual del programador con los cambios realizados hasta
    este punto.
    \item Añadir requisitos de la apliación web.
    \item Añadir diagrama general de casos de uso.
    \item Cambiar la estructura de directorios del frontend de la web.
    \item Añadir y configurar contenedor de base de datos PostgreSQL al proyecto
    de Docker.
    \item Configurar la API para interactuar con la base de datos, creando, los
    modelos de SQLAlchemy con los que se va a trabajar.
    \item Añadir algunos \textit{endpoints} para permitir una gestión muy básica de
    usuarios.
    \item Bloquear la versión de los contenedores de Docker para evitar
    problemas de dependencias en el futuro.
    \item Añadir persistencia de datos entre creaciones del contenedor de la
    base de datos.
    \item Bloquear versiones de las dependencias de Python. Debido a la versión
    2 de Pandas, se rompe parte de la funcionalidad.
    \item Implementar una versión inicial del \textit{endpoint} para obtener una
    predicción a partir de un vídeo.
\end{itemize}

\subsection{Sprint 10 (10/4/2023 -- 16/4/2023)}

En este sprint se realizaron varias tareas sobre la aplicación web para permitir
la fácil interacción con el formulario para obtener predicciones.

Tareas realizadas:

\begin{itemize}
    \item Añadir y estilizar un formulario básico para interactuar desde el
    frontend con la API y obtener un predicción.
    \item Utilizar JavaScript para enviar los contenidos de este formulario de
    forma correcta al \textit{endpoint}.
    \item Añadir barra de progreso de subida del vídeo una vez el formulario es
    enviado.
    \item Implementar el entrenamiento del modelo final con todos los datos y
    añadirlo a la API para poder realizar la predicción.
    \item Añadir funcionalidad de arrastrar y soltar al formulario para subir el
    vídeo.
\end{itemize}

\subsection{Sprint 11 (17/4/2023 -- 23/4/2023)}

En este sprint se internacionalizó la aplicación web, entre otras cosas de menor
importancia.

Tareas realizadas:

\begin{itemize}
    \item Crear prototipos (mockups) para la página principal y para la gestión
    de modelos.
    \item Internacionalizar la aplicación mediante \texttt{TypeSafe-i18n} y
    añadir un selector de idioma.
    \item Añadir autenticación mediante tókenes JWT (JSON Web Token) a la API.
\end{itemize}

\subsection{Sprint 12 (24/4/2023 -- 30/4/2023)}

En este sprint se solucionaron algunos bugs y mejoró la accesibilidad de la
apliación web en gran medida gracias a un
\href{https://www.udacity.com/course/web-accessibility--ud891}{curso sobre
accesibilidad} que se realizó.

Tareas realizadas:

\begin{itemize}
    \item Implementar la autenticación de usuario con la API desde la página web.
    \item Refactorizar y limpiar el código de la web.
    \item Solucionar bug de idioma indefinido.
    \item Pulir los estilos de algunos elementos de la web.
    \item Solucionar bug pantalla de administración se muestra brevemente antes
    de login.
    \item Mejorar accesibilidad del HTML.
    \item Añadir mensajes acompañando a la predicción dependiendo del resultado.
\end{itemize}

\subsection{Sprint 13 (1/5/2023 -- 7/5/2023)}

En este sprint se implementó la administración de usuarios desde la página web.

\begin{itemize}
    \item Pulir la pantalla de resultados de la web.
    \item Añadir ejecución de un script SQL por defecto en la creación de la
    base de datos.
    \item Añadir \textit{endpoints} para crear y eliminar usuarios en la API.
    \item Añadir página de gestión de usuarios.
\end{itemize}

\subsection{Sprint 14 (8/5/2023 -- 14/5/2023)}

En este sprint se implementó la administración de modelos y añadió algunos
apartados de la documentación.

\begin{itemize}
    \item Documentar conceptos teóricos y planificación.
    \item Implementar administración de modelos.
    \item Añadir pestañas para navegar entre administración de usuarios y de
    modelos.
    \item Iniciar pruebas con optimización bayesiana.
    \item Añadir aviso para la predicción.
    \item Añadir introducción a memoria.
\end{itemize}

\subsection{Sprint 15 (15/5/2023 -- 21/5/2023)}

En este sprint se pulieron algunos aspectos sobre la administración de modelos y
continuó con la documentación.

\begin{itemize}
    \item Añadir objetivos a la memoria.
    \item Arreglar bug botón <<Administración>>.
    \item Añadir validación de los modelos subidos.
    \item Añadir licencia.
    \item Documentar viabilidad económica y legal.
\end{itemize}

\subsection{Sprint 16 (22/5/2023 -- 28/5/2023)}

En este sprint se realizaron pequeños arreglos sobre la web y la documentación.

\begin{itemize}
    \item Realizar pequeños arreglos sobre la web.
    \item Reorganizar herramientas en la memoria.
\end{itemize}

\subsection{Sprint 17 (29/5/2023 -- 04/6/2023)}

En este sprint se continuó con la documentación y creó el <<script>> de
generación de modelos.

\begin{itemize}
    \item Documentar aspectos relevantes sobre el proyecto.
    \item Añadir comentarios de documentación a funciones y clases del proyecto.
    \item Añadir <<script>> de generación de modelos.
    \item Documentar estuctura arquitectónica.
\end{itemize}

\section{Estudio de viabilidad}

En este apartado se analizan algunos aspectos que se deberían tener en cuenta si
el producto creado en este proyecto fuese llevado a un entorno más real, en el
que se tuviese que responder ante otras personas si hubiese algún problema.

\subsection{Viabilidad económica}

En este subapartado se va a realizar un pequeño estudio económico sobre los
costes y las ganancias de este proyecto.

\subsubsection{Costes}

Existen varios costes, tanto fijos como variables, que tendría esta aplicación

\paragraph{Empleados}

Durante el transcurso del desarrollo del proyecto se han invertido
aproximadamente 450 horas de trabajo durante un periodo de 5 meses.

El salario medio en españa de un programador junior es de unos 22.000€, como de
media se trabaja al rededor de 1.826 horas al año, esto equivale a unos 12€ por
hora.

Con estas cifras se obtiene el siguiente salario mensual bruto.

$$
450\text{ horas} \times 12 \frac{\textup{€}}{hora} \text{ / } 5\text{ meses} = 1.080 \frac{\textup{€}}{mes}
$$

Para obtener el gasto que supone este salario se deben añadir los impuestos que
corren a cargo de la empresa, estos son los siguientes según
\href{https://www.seg-social.es/wps/portal/wss/internet/Trabajadores/CotizacionRecaudacionTrabajadores/36537?changeLanguage=es}{la
Seguridad Social}:

\begin{itemize}
      \item \textbf{Contingencias comunes}: Para tratar alteraciones de la salud
      no relacionadas con un accidente de trabajo o enfermedad profesional.
      Corresponde con el 23,60\% del salario.
      \item \textbf{Desempleo}: Correspondiente al 5,50\%.
      \item \textbf{FOGASA} (Fondo de Garantía Salarial): Para abonar a los
      trabajadores el importe de salarios pendientes por diversas razones
      relacionadas con la quiebra de una empresa. Corresponde al 0,20 \%.
      \item \textbf{Formación profesional}: Utilizado para incentivar la
      formación de trabajadores. Corresponde al 0,60\%.
\end{itemize}

Según los impustos anteriores el importe a pagar por la empresa en materia de
salario es:

$$
\frac{900\frac{\textup{€}}{mes}}{1-(0.236+0.055+0.002+0.006)} = 1283.88 \frac{\textup{€}}{mes}
$$

\paragraph{Hardware}

\subparagraph{Equipo para el desarrollo}

Para el desarrollo de la aplicación se han utilizado dos equipos informáticos,
uno de escritorio y otro portátil, con un precio de alrededor de 1100€ y
1200€ respectivamente.

Estableciendo un periodo de amortización de 4 años para ambos bienes se obtiene
un coste anual de:

$$
\frac{1100\textup{€} + 1200\textup{€}}{4\text{ }años} = 575 \frac{\textup{€}}{año}
$$

Como el proyecto ha sido realizado durante 5 meses la amortización total en este
periodo de tiempo es de 239,58€.

\subparagraph{Alojamiento de la aplicación}

La aplicación debe ser alojada en una máquina con prestaciones suficientes como
para cumplir con los requisitos funcionales y no funcionales establecidos.

Una opción sería alquilar espacio en un centro de datos de alojamiento en el que
instalar un equipo informático propio, esto tiene ciertas ventajas al tener
completo control sobre los componentes, pero al mismo tiempo supone un gasto muy
grande, ya que se debe tener en cuenta el gasto inicial de adquisición del
hardware que podría rondar entre 500€ y 1000€, además del gasto recurrente de
alquiler del alojamiento que, para un \textit{rack} de una unidad suele estar en
torno a los 100€ mensuales.

Otra opción, y por la que se ha decidido al final, es utilizar alguno de los
servicios de alojamiento virtualizados ofrecidos por varios proveedores como
Hetzner o OVH. Se han barajado ambas opciones y, aunque OVH dispone de una red
con un servicio más rápido para España, Hetzner ofrece una mejor relación
rendimiento/precio, por lo que el servicio utilizado para alojar la aplicación
es el servidor virtualizado CPX21 (3 procesadores virtuales, 4GB de RAM, 80GB de
almacenamiento, IPv4 y tráfico de subida de 20TB) de Hetzner, con un precio de
9,13€ por mes.

La aplicación ha sido alojada durante los últimos 3 meses, por lo que el coste
total de esto ha sido de:

$$
9,13 \frac{\textup{€}}{mes} \times 3 \text{meses} = 27,39\textup{€}
$$

Todo lo anterior resulta en unos costes totales por hardware de 266,97€.

\paragraph{Software}

La mayor parte de programas utilizados durante el desarrollo son gratuitos, la
única excepción es la versión profesional del IDE PyCharm, que aunque ha sido
gratuito gracias a la licencia para estudiantes, en un entorno empresarial esto
no sería posible. El precio mensual de este producto es de 30,13€. Por lo que el
coste total durante estos 5 meses sería de:

$$
5\text{ meses} \times 30,13 \frac{\textup{€}}{mes} = 150,65\textup{€}
$$

\paragraph{Costes totales}

La siguiente tabla muestra de forma resumida los costes en los que se ha
incurrido durante el desarrollo del proyecto.

\tablaSmallSinColores{Costes}{l r}{costes}
{ Concepto & Coste (€) \\}{ 
Empleados & 1283,88 \\
Hardware & 266,97 \\
Software & 150,65 \\
\toprule
\textbf{Total} & \textbf{1701,50} \\
}

\subsubsection{Ganancias}

Uno de los objetivos del proyecto es que el uso del sistema sea accesible, por
lo que imponer una barrera monetaria iría directamente en contra de este
objetivo. Por lo que se ha decidido que la aplicación web va a ser gratuita.
Debido a esto, la empresa no obtendrá beneficios. Esto se podría solventar en
cierto modo abriendo un portal de donaciones para intentar cubrir al menos los
costes de alojamiento.

\subsection{Viabilidad legal}

La siguiente tabla recopila las licencias de las diferentes librerías y
utilidades de las que depende la aplicación junto con las licencias bajo las que
han sido distribuidas.

\tablaSmall{Resumen de licencias de las bibliotecas/herramientas utilizadas}{c c}{licencias}
{ Librería & Licencia \\}{ 
numpy & 3-Clause BSD License \\
pandas & 3-Clause BSD License \\
scipy & 3-Clause BSD License \\
sklearn & 3-Clause BSD License \\
imbalanced-learning & MIT \\
pydantic & MIT \\
Mediapipe & Apache License 2.0 \\
OpenCV Python & MIT \\
tsfresh & MIT \\
matrixprofile & Apache License 2.0 \\
xgboost & Apache License 2.0 \\
skopt & 3-Clause BSD License \\
fastapi & MIT \\
uvicorn & 3-Clause BSD License \\
sqlalchemy & MIT \\
psycopg2 & GNU Lesser General Public License \\
python-multipart & Apache License 2.0 \\
python-jose & MIT \\
passlib & 2-Clause BSD License \\
caddy & Apache License 2.0 \\
nodejs & MIT \\
sveltekit & MIT \\
tailwindcss & MIT \\
axios & MIT \\
resolve-accept-language & MIT \\
typesafe-i18n & MIT \\
PostgreSQL & \href{https://opensource.org/license/postgresql/}{PostgreSQL Licence} (Similar a BSD y MIT) \\
}

Todas las librerías y herramientas utilizadas tienen licencias libres típicas en
programas de código abierto que permiten su uso de forma gratuita. Gracias a
esto la aplicación puede ser distribuida bajo casi cualquier otra licencia, en
este caso se ha optado por utilizar la licencia MIT, que otorga a cualquier
persona que obetenga una copia del proyecto a usarlo, modificarlo y
redistribuirlo libremente, al mismo tiempo que el dueño está protegido de la
responsabilidad legal de cualquier problema que pueda dar el uso de dicho
programa.
