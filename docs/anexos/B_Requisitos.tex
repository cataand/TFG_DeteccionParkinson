\newcounter{ncu}
\setcounter{ncu}{0}
\newcommand{\cu}[9]{
    \addtocounter{ncu}{1}

    \begin{table}[p]
        \centering
        \begin{tabularx}{\linewidth}{ p{0.21\columnwidth} p{0.71\columnwidth} }
            \toprule
            \textbf{CU-\arabic{ncu}}      & \textbf{#1} \\
            \toprule
            \textbf{Versión}              & #2          \\
            \textbf{Autor}                & \theauthor  \\
            \textbf{Requisitos asociados} & #3          \\
            \textbf{Descripción}          & #4          \\
            \textbf{Precondición}         & #5          \\
            \textbf{Acciones}             &
            #6                                          \\
            \textbf{Postcondición}        & #7          \\
            \textbf{Excepciones}          & #8          \\
            \textbf{Importancia}          & #9          \\
            \bottomrule
        \end{tabularx}
        \caption{CU-\arabic{ncu} #1.}
    \end{table}
}


\capitulo{Especificación de Requisitos}
\label{cha:Especificación de Requisitos}

\section{Introducción}

Antes de comenzar a crear un programa es importante detallar las características
que deberá tener (quienes son los usuarios, qué acciones pueden realizar, de qué
manera deberá responder el programa, etc.). En este caso se ha seguido UML
(Unified Modelling Language) como estándar para definir estas características.

\section{Objetivos generales}

\section{Catálogo de requisitos}

\subsection{Requisitos funcionales}

\begin{enumerate}[start=1,label={\bfseries RF\arabic*:}]
    \item El sistema debe permitir diferenciar entre dos roles: Usuario normal y
          Administrador.
    \item Cualquier usuario debe poder subir un archivo de vídeo junto con
          información adicional para ser procesados por un modelo de
          inteligencia artifical y obtener el resultado.
    \item Los usuarios administradores deben poder iniciar sesión con su nombre
          de usuario y contraseña respectivos.
    \item Los usuarios administradores deben poder finalizar su sesión.
    \item Un inicio de sesión debe ser mantenido entre cambios de página y
          cierres del navegador durante 24 horas.
    \item Un administrador debe poder gestionar una lista con los modelos
          disponibles.
    \item Un administrador debe poder añadir modelos a la lista de modelos
          disponibles.
    \item Un administrador debe poder eliminar modelos de la lista de modelos
          disponibles.
    \item Un administrador debe poder seleccionar de la lista de modelos
          disponibles aquel que se va a utilizar para realizar predicciones.
\end{enumerate}

\subsection{Requisitos no funcionales}

\begin{enumerate}[start=1,label={\bfseries RNF\arabic*:}]
    \item El tiempo desde que un usuario sube un vídeo y sus datos hasta que
          recibe el resultado debe ser inferior a un minuto.
    \item El sistema debe ser fácil de usar, con una interfaz intuitiva que el
          usuario entienda de forma inmediata.
    \item La página web deberá utilizar etiquetas y atributos HTML de tal forma
          que se facilite la navegación para personas con accesibilidad
          limitada.
    \item La información de los usuario se debe almacenar de forma segura,
          cifrando campos pertinentes como la contraseña.
\end{enumerate}

\section{Especificación de requisitos}

\subsection{Actores}

Existen dos tipos de actores en esta aplicación:

\begin{itemize}
    \item \textbf{Usuario Normal}: Es aquel usuario que no ha iniciado sesión y
          tiene un acceso limitado a las capacidades del sistema.
    \item \textbf{Administrador}: Es aquel usuario que sí ha iniciado sesión en
          la aplicación y tiene acceso a todas las funcionalidades.
\end{itemize}

\subsection{Casos de uso}

\imagen{anexos/diagrama_casos_de_uso.pdf}{Diagrama general de casos de uso}{0.5}

\cu{Obtener predicción}
{1.0}
{RF2}
{Cualquier usuario debe poder obtener un predicción utilizado un modelo.}
{Estar en la página respectiva de la web.}
{
    \begin{enumerate}
        \def\labelenumi{\arabic{enumi}.}
        \tightlist
        \item El usuario selecciona el archivo de vídeo sobre el que desea
              recibir una predicción.
        \item El usuario introduce la información adicional necesaria.
        \item El usuario hace clic sobre el botón para subir la infomación.
        \item Se muestra una pantalla que informa al usuario sobre el estado de
              su petición.
        \item Se muestra la predicción obtenida.
    \end{enumerate}
}
{}
{
    \begin{itemize}
        \item [1] No se pueden extraer fotogramas con pose suficientes.
        \item [2] La información adicional introducida no es válida.
    \end{itemize}
}
{Alta}

\cu{Iniciar sesión}
{1.0}
{RF1, RF3}
{Los usuarios administradores deben poder iniciar sesión con sus credenciales.}
{Estar en la página de inicio de sesión.}
{
    \begin{enumerate}
        \def\labelenumi{\arabic{enumi}.}
        \tightlist
        \item El usuario introduce su nombre de usuario.
        \item El usuario introduce su contraseña.
        \item El usuario hace clic en el botón de ``iniciar sesión''.
    \end{enumerate}
}
{El usuario es redirigido a la página de administración pertinente.}
{
    \begin{itemize}
        \item [3] Las credenciales introducidas no son válidas.
    \end{itemize}
}
{Alta}

\cu{Finalizar sesión}
{1.0}
{RF1, RF4}
{Los usuarios administradores deben poder finalizar su sesión.}
{Tener una sesión activa.}
{
    \begin{enumerate}
        \def\labelenumi{\arabic{enumi}.}
        \tightlist
        \item El usuario hace clic en el botón de la parte superior para cerrar su sesión.
        \item La sesión se cierra y el usuario es redirigido a la página principal.
    \end{enumerate}
}
{}
{}
{Media}

\cu{Ver lista de modelos}
{1.0}
{RF1, RF6}
{Los usuarios administradores deben poder ver la lista con los modelos disponibles.}
{}
{
    \begin{enumerate}
        \def\labelenumi{\arabic{enumi}.}
        \tightlist
        \item El usuario accede a la página de gestión de modelos.
        \item Se muetra una lista con los modelos existentes junto con algunas acciones.
    \end{enumerate}
}
{}
{
    \begin{itemize}
        \item [1] Usuario no ha iniciado sesión.
    \end{itemize}
}
{Alta}

\cu{Añadir modelo}
{1.0}
{RF1, RF7}
{Los usuarios administradores deben poder añadir modelos.}
{El usuario debe estar en la lista de modelos.}
{
    \begin{enumerate}
        \def\labelenumi{\arabic{enumi}.}
        \tightlist
        \item El usuario hace clic en el botón para añadir un modelo.
        \item Se abre un formulario donde se pide el archivo del modelo junto
              con información adicional que describa al modelo.
        \item El usuario rellena el formulario y sube el archivo.
        \item Se muestra una página de carga mientras se procesa la información.
        \item Se devuelve al usuario a la lista de modelos, dondel el nuevo
              modelo debería estar presente.
    \end{enumerate}
}
{}
{
    \begin{itemize}
        \item [1] Usuario no ha iniciado sesión.
        \item [5] Modelo subido no válido.
    \end{itemize}
}
{Alta}

\cu{Eliminar modelo}
{1.0}
{RF1, RF8}
{Los usuarios administradores deben poder eliminar modelos.}
{El usuario debe estar en la lista de modelos.}
{
    \begin{enumerate}
        \def\labelenumi{\arabic{enumi}.}
        \tightlist
        \item El usuario hace clic en el botón para eliminar un modelo en concreto.
        \item Se abre una ventana modal que pregunta al usuario si está seguro.
        \item Si está seguro, se refresca la página mostrando la información actualizada.
    \end{enumerate}
}
{
    El modelo ha sido eliminado del disco donde estaba almacenado y su entrada en
    la base de datos ha sido eliminada.
}
{
    \begin{itemize}
        \item [1] Usuario no ha iniciado sesión.
    \end{itemize}
}
{Alta}

\cu{Seleccionar modelo}
{1.0}
{RF1, RF9}
{Los usuarios administradores deben poder seleccionar el modelo a usar para realizar predicciones.}
{El usuario debe estar en la lista de modelos.}
{
    \begin{enumerate}
        \def\labelenumi{\arabic{enumi}.}
        \tightlist
        \item El usuario hace clic en el botón para seleccionar un modelo
              concreto de la lista de modelos.
        \item Se informa al usuario de que el modelo ha sido seleccionado y es
              el que se va a utilizar para realizar predicciones.
    \end{enumerate}
}
{
    La base de datos cambia para establecer el modelo respectivo como seleccionado.
}
{
    \begin{itemize}
        \item [1] Usuario no ha iniciado sesión.
    \end{itemize}
}
{Alta}
